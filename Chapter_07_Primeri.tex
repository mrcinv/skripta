\chapter{Primeri laserjev}

Danes je v rabi veliko "stevilo razli"cnih laserjev in tu ne moremo opisati
vseh. Na kratko poglejmo le nekaj najpomembnej"sih, ne da bi se pri tem
spu"s"cali v podrobnosti in razli"cne mo"zne izvedbe. Navadno laserje
razlikujemo po aktivnem sredstvu, pri "cemer pa pogosto obstoja mnogo
razli"cnih izvedb. Eno sredstvo lahko slu"zi na primer tako v zvezno
delujo"cem laserju kot v takem, ki deluje le v kratkih sunkih.

Laser je lahko dokaj preprosta naprava, z malo sestavnimi deli, lahko pa je
tudi zelo velik in zapleten sistem. Ve"cina velikih laserskih sistemov, ki
dajejo zelo veliko svetlobno mo"c, je sestavljenih iz osnovnega laserja, ki
ni posebno mo"can, daje pa kvalitetno svetlobo, in iz enega ali ve"c
oja"cevalnikov. V njih se svetloba oja"cuje v sredstvu, ki je enako kot v
osnovnem laserju in ki je v kolikor mogo"ce visokem stanju obrnjene
zasedenosti, le da brez resonatorja. V ve"cih oja"cevalnih stopnjah je tako
mo"zno dose"ci zelo veliko svetlobno mo"c. Pri tem nastopi tudi vrsta novih
problemov. Da gostota svetlobnega toka ni prevelika in ne povzro"ca po"skodb
opti"cnih komponent, mora premer oja"cevanega snopa in s tem premer
oja"cevalnih stopenj nara"s"cati. Zadnje stopnje velikega laserskega sistema
v Rochestru imajo na primer premer pol metra, kar seveda pomeni, da morajo
imeti tolik"sno opdrtino tudi vse ostale opti"cne komponente. Polega tega je
potrebno skrbno paziti, da se odbita svetloba ne vra"ca v prej"snji
oja"cevalnik ali v osnovni laser in moti njegovo delovanje. Zato so med
oja"cevalnimi stopnjami opti"cni izolatorji, ki temeljijo na Faradayevem
pojavu vrtenja polarizacije v snovi v magnetnem polju.

V laserske resonatorje pogosto vgrajujemo tudi dodatne opti"cne elemente, s
katerimi vplivamo na delovanje laserja. V prej"snjem poglavju smo "ze
govorili o modulatorjih za spreminjanje dobrote, s "cemer dobimo mo"cne
kratke sunke ali uklenemo faze ve"cih lastnih nihanj. Dobroto resonatorja
lahko spreminjamo tako v laserju, ki ga "crpamo le v sunkih, kot v stalno
"crpanem laserju. V drugem primeru deluje laser neprekinjeno, kadar je
modulator izklju"cen, ali v sunkih, "ce modulator deluje. V nekaterih
laserjih sta celo dva modulatorja, eden za preklaplanje dobrote, drugi za
uklepanje faz, ki lahko delujeta vsak posebej ali oba hkrati. V zadnjem
primeru dobimo iz laserja ob vsakem preklopu kvalitete kratko zaporedje
fazno uklenjenih sunkov (Slika \ref{6.1}).

V na"sem pregledu bomo navedli le osnovne karakteristike najpomebnej"sih
laserjev in se ne bomo spu"s"cali v podrobnosti razli"cnih izvedb.

\section{Neodimov laser}

Eden najpomembnej"sih laserjev je osnovan na itrij- aluminijevem granatu Y$_3
$Al$_5$O$_{12}$ (YAG) s primesnimi ioni Nd$^{3+}$. Valovna dol"zina, pri
kateri neodimov laser deluje, je 1,06 $\mu$m. Redke zemlje, med katere sodi
neodim, imajo zelo bogat spekter vzbujenih stanj v opti"cnem podro"cju. Del
ga ka"ze slika \ref{6.2}. Zna"cilno za spektre ionov redkih zemelj, ki so
vgrajeni v neko kristalno mre"zo, je, da so le malo moteni zaradi vpliva
kristalnega polja, to je elektrostatske interakcije z okoli"skimi atomi.
Nezapolnjena $4f$ orbitala je namre"c deloma sen"cena z zapolnjeno $6s$
orbitalo. Zato so nivoji v kristalu podobni kot za prost atom in so le malo
raz"sirjeni zaradi termi"cnih motenj okolice.

Neodimov laser deluje med stanjema $^4$F$_{3/2}$ in $^4$I$_{11/2}$. "Sirina
tega prehoda je 180~GHz. Presek za stimulirano sevanje, ki smo ga definirali
v razdelku 4.4, je $\sigma=9\times 10^{-19}$~cm$^2$, kar je precej velika
vrednost. Poleg tega je "zivljenski "cas gornjega laserskega stanja $%
1/A=5.5\times 10^{-4}$~s, spodnjega pa mnogo manj. Zato je lahko dose"ci
veliko zasedenost gornjega stanja in veliko obrnjeno zasedenost. Spodnje
stanje je dovolj visoko nad osnovnim, da pri sobni temperaturi v ravnovesju
ni znatno zasedeno. Zaradi vsega tega je prag neodimovega laserja zelo nizek
in je prav lahko dose"ci zvezno stacionarno delovanje, prav tako dobro pa
deluje tudi v sunkih.

Kot je obi"cajno za laserje, v katerih je aktivna snov kristal ali steklo s
primesmi, neodimov laser "crpamo s svetlobo vi"sje frekvence, kot je
laserski prehod. Najpomembnej"si absorpcijski pasovi so tik pod vidnim
podro"cjem pri 790~nm in 750~nm in v zelenem delu vidnega spektra. Aktivna
snov je v obliki pali"cke dol"zine od nekaj cm do dobrih 10~cm. Za "crpanje
uporabljamo mo"cne ksenonove svetilke za zvezno delovanje ali podobne
bliskovne lu"ci za sunkovno delovanje. Skupaj z aktivno pali"cko so vgrajene
v cilindri"cno ali elipti"cno z zrcalnimi ali belimi stenami, tako da se
"cim ve"cji del "crpalne svetlobe absorbira v laserski pali"cki (Slika \ref
{s6.1}).

Pri "crpanju s ksenonovo lu"cjo je le manj"si del "crpalne svetlobe v
absorpcijskih pasovih, zato je izkoristek dokaj slab, pod 0.01. Preostala
mo"c gre v gretje, zaradi "cesar je v laserjih z nekoliko ve"cjo povpre"cno
mo"cjo potrebno vodno hlajenje. Gretje povzro"ca tudi toplotne deformacije
laserske pali"cke, kar lahko mo"cno spremeni lastnosti resonatorja. Toplotni
u"cinki so ena poglavitnih prakti"cnih te"zav pri izdelavi neodimovih
laserjev.

V zadnjem "casu se uveljavlja tudi "crpanje z mno"zico polvodni"skih diodnih
laserjev, ki svetijo v pasu med 750~nm in 800~nm, to je ravno v obmo"cju
absorpcije Nd. Zato je izkoristek dosti bolj"si in je gretja manj, kar
omogo"ca bolj kompaktno konstrukcijo, bolj"so stabilnost izhodne mo"ci in
ve"cjo zanesljivost.

Naloga: Oceni iz gornjih podatkov, kolik"sna je potrebna mo"c ksenonove
svetilke, da dose"zemo prag delovanja neodimovega laserja z dol"zino
resonatorja 20~cm in reflektivnostjo izhodnega zrcala 0.96. Ksenonova
svetilka pretvori v svetlobo pribli"zno polovico elektri"cne mo"ci.

Tipi"cna izhodna mo"c zvezno delujo"cega Nd-YAG laserja je nekaj deset W, za
kar je potrebna elektri"cna mo"c nekaj kW.

Pri delovanju v sunkih se skozi bliskovno lu"c izprazni nabit kondenzator,
ki mora imeti precej veliko kapacitivnost, da je energija bliska zadostna,
to je vsaj nekaj J. "Cas trajanja bliska je dolo"cen RC konstanto
kondenzatorja in lu"ci in je okoli 0.1~ms. Tipi"cna energija izhodnega sunka
je okoli 0.1~J.

Namesto v ustrezen kristal je Nd lahko vgrajen tudi v steklo. Zaradi
neurejene okolice je laserska "crta precej "sir"sa, blizu 1 THz in
"zivljenski "cas gornjega nivoja malo kraj"si, okoli 0.3~ms. Zato je
oja"cenje manj"se kot v Nd-YAG in je za prag laserskega delovanja potrebna
precej ve"cja mo"c "crpanja. Laserji Nd-steklo zato delujejo le v sunkovnem
na"cinu, kjer pa so za velike energije celo bolj"si od Nd-YAG. Zaradi
manj"sega oja"cenja pri dani obrnjeni zasedenosti je v laserju s preklopom
kvalitete mo"zno dose"ci ve"cjo na"crpanost, ne da bi pri"slo do praznenja
zaradi oja"cevanja spontanega sevanja v enem preletu pali"cke. Energija
izhodnih sunkov laserjev Nd- steklo so tako do nekaj J.

Ve"cje energije sunkov je mogo"ce dobiti z oja"cevalniki. Med najve"cjimi je
laserski sistem Nd-steklo v Rochestru v dr"zavi New York, ki ga uporabljajo
za raziskave fuzije. Okoli 1~ns dolg sunek iz osnovnega laserja razdelijo na
deset oja"cevalnih vej, ki so dolge po 180~m. Da prepre"cijo po"skodbe
povr"sin elementov zaradi prevelike gostote svetlobne energije, morajo snope
raz"siriti, tako da je premer zadnjih oja"cevalnih stopenj pol metra.
Kon"cna energija sunka je nad 100~kJ. Z njim z vseh strani posvetijo na
kroglico iz devterija in tritija, ki se dovolj segreje in stisne, da pride
do zlivanja devterija in tritija. Vr"sna mo"c laserskega sunka je $10^{13}$%
~W. "Ce jo zberemo na povr"sino 1~mm$^2$, dobimo elektri"cno poljsko jakost
okoli 10$^{11}$~V/m$^2$, kar je pribli"zno enako polju v vodikovem atomu.

\section{He-Ne laser}

Prvi zvezno delujo"ci laser je bil He-Ne laser, v katerem je prehod med
stanji 2S in 2p atomov Ne dal svetlobo z valovno dol"zino 1.15~$\mu$m.
Prehodi v Ne dajo lasersko delovanje tudi pri 3.39~$\mu$m, 632.8~nm in nekaj
valovnih dol"zinah v oran"znem in zelenem delu spektra. Energijska stanja He
in Ne ka"ze slika \ref{s6.2}. Najpomembnej"si je prehod pri 632.8~nm.

Slika \ref{s6.3} ka"ze shemo tipi"cnega plinskega laserja, kakr"sen je He-Ne
laser. Elektri"cni tok te"ce skozi razelektritveno cev, v kateri je
me"sanica pribli"zno 1~mb He in 0.1~mb Ne. Okna cevi so nagnjena za
Brewsterjev kot, da so izgube pri odboju za eno polarizacijo "cim manj"se.
Izhodna svetloba iz laserja je zato seveda polarizirana. V manj"sih laserjih
so namesto Brewsterjevih oken na razelektritveno cev privarjena kar
resonatorska zrcala. Tak laser je nepolariziran. Elektroni, ki so glavni
nosilci toka, s trki vzbujajo atome He v ve"c vzbujenih stanj. Ti se
vra"cajo v osnovno stanje, pri tem pa se nabirajo v dolgo"zivih
metastabilnih stanjih 2$^3$S in 2$^1$S, ki imata razpadne "case 0.1~ms in 5~$%
\mu$s. Ti dve stanji imata skoraj enako energijo kot 2S in 3S stanja Ne in
se energija pri trkih vzbujenih He atomov z Ne atomi v osnovnem stanju lahko
prenese na Ne atome. Majhna energijska razlika (okoli 1.2 THz) preide v
kineti"cno energijo obeh atomov. Ta prenos je glavni "crpalni proces v He-Ne
laserju.

Znano rde"co svetlobo He-Ne laserja dobimo pri prehodu s 3S stanja na eno od
2p stanj. "Zivljenski "cas 3S stanja je 10$^{-7}$~s, 2p stanje pa v okoli 10$%
^{-8}$~s preide s sevanjem v 1S stanje. To je metastabilno - dipolni sevalni
prehodi v osnovno stanje so prepovedani, zato se v njem atomi nabirajo in se
pri trkih z elektroni vra"cajo v spodnje lasersko stanje 2p. To zmanj"suje
obrnjeno zasedenost. Atomi preidejo z 1S stanja v osnovno najve"c pri trkih
s steno cevi, zato oja"cenje raste z zmanj"sevanjem premera cevi.

Lasersko delovanje dobimo tudi pri prehodu s 3S na 3p, ki ima valovno
dol"zino 3.39~$\mu$m. Oja"cenje je za ta prehod je celo precej ve"cje kot za
632.8~nm, deloma zaradi ni"zje frekvence (glej zvezo med Einsteinovima
koeficientoma A in B), deloma pa zaradi kratke "zivljenske dobe spodnjega
laserskega nivoja 3p. Zato bi pri"cakovali, da bo He-Ne laser svetil pri
3.39~$\mu$m in ne pri 632.8~nm. To prepre"ci absorpcija v steklu in
selektivna reflektivnost resonatorskih zrcal, kar dvigne prag za 3.39~$\mu$m
nad prag za 638.2~nm.

Izhodna mo"c He-Ne laserja je od nekaj desetink mW do 100~mW. So preprosti,
zanesljivi in poceni, zato so poleg polvodni"skih najve"c uporabljani
laserji. Uporabljamo jih merilnih napravah, v opti"cnih "citalnih sistemih,
na primer v "citalcih "crtne kode, v "solah, v raziskovalnih laboratorijih
za intereferometrijo itd. Na njem je osnovan tudi sekundarni standard za
meter, kot smo videli v prej"snjem poglavju.

\section{Argonski ionski laser}

Prehodi v ioniziranem Ar omogo"cajo lasersko delovanje v pri vrsti valovnih
dol"zin v zelenem, modrem in bli"znjem UV podro"cju spektra. Zato je Ar
ionski laser med najpomembnej"simi, ki se danes porabljajo. Kot ve"cino
drugih plinskih laserjev tudi tega "crpamoz elektri"cnim tokom. Da dobimo
ioniziran argon z obrnjeno zasedenostjo in dovolj velikim oja"cenjem, mora
biti elektri"cni tok precej velik, nekaj deset amperov. S tem je velika tudi
potrebna elektri"cna mo"c, do 10 kW in ve"c. Zaradi velike koli"cine
odve"cne toplote je navadno Ar laser vodno hlajen.

Ar laser da najve"cjo svetlobno mo"c pri 488 nm in 514.5~nm, do kakih 20 W.
Da izberemo eno od mo"znih "crt, moramo v resonator vgraditi "se nek
frekven"cno selektiven element. Najpogosteje uporabijo kar majhno prizmo
pred enim od obeh zrcal, kot ka"ze slika \ref{s6.3a}. Z vrtenjem sklopa
prizme in zrcala lahko izberemo valovno dol"zino svetlobe, ki je pravokotna
na obe zrcali.

Argonski laser je zanesljiv in lahko daje zelo kvaliteten izhodni snop, to
je, brez te"zav dose"zemo, da deluje v osnovnem Gaussovem na"cinu in pri eni
sami frekvenci. Zato se dosti uporablja v opti"cni spektroskopiji,
interferometriji, holografiji in merilni tehniki. Zelo podoben mu je "se
kriptonov laser, ki deluje v rde"cem in oran"znem delu spektra.

\section{Laser na ogljikov dioksid}

Vsi doslej opisani laserji so delovali na elektronskih prehodih. CO$_2$
laser pa izrablja prehode med vibracijskimi stanji molekule. Pri tem
elektroni ostanejo v osnovnem stanju. Energije vibracijskih prehodov so 10
do 100 krat manj"se od elektronskih, zato CO$_2$ in drugi podobni
molekularni laserji delujejo v infrarde"cem podro"cju.

Molekula CO$_2$ je prikazana na sliki \ref{s6.4}. V osnovnem stanju (a) je
linearna. Atomi lahko nihajo glede na te"zi"s"ce na tri na"cine: oba kisika
se gibljeta simetri"cno vzdol"z osi molekule, pri "cemer ogljik miruje -
simetri"cni razteg (b), atomi nihajo simetri"cno v smeri pravokotno na os -
upogib (c) in atoma kisika se gibljeta oba v isti smeri vzdol"z osi, ogljik
pa v nasprotni smeri - asimetri"cni razteg (d). Najvi"sjo frekvenco ima
asimetri"cni razteg, najni"zjo pa upogib. Energija vsakega nihanja je dana s
"stevilom energijskih kvantov v nihanju; celotna nihajna energija molekule CO%
$_2$je torej podana s tremi celimi "stevili $(n_1,n_2,n_3)$.

Nekaj najni"zjih vibracijskih stanj molekule CO$_2$ je prikazanih na sliki 
\ref{s6.5}. CO$_2$ laser vzbujamo z elektri"cnim tokom skozi me"sanico CO$_2$
in du"sika. Podobno kot v He-Ne laserju se tudi CO$_2$ "crpa predvsem preko
trkov z du"sikovimi molekulami, katerih prvo vzbujeno vibracijsko stanje ima
skoraj enako energijo kot stanje (001), ki je gornje stanje CO$_2$ laserja.
Laserski prehod med stanjema (001) in (100) ima valovno dol"zino 10.6 $\mu$%
m. Celotni izkoristek laserja je razmeroma velik, blizu 30%
%. Poleg tega lahko dajejo tudi precej veliko mo"c, preko
10 kW v stalnem delovanju. Poleg velikega izkoristka k temu prispeva tudi
to, da se molekule iz spodnjega laserskega stanja hitro vrnejo v osnovno
stanje, kjer jih je mo"c zopet porabiti v oja"cevalnem procesu. To se zgodi
predvsem preko trkov z drugimi molekulami ali atomi, na primer He, ki je
dodan plinski me"sanici.

CO$_2$ laserji se uporabljajo najve"c v industriji za zahtevne obdelave
materialov, na primer za rezanje plo"cevine po krivih robovih. Obdelava z
laserji omogo"ca veliko natan"cnost in "cisto"co in je zelo fleksibilna.

\section{Ekscimerni laser}

Ekscimerji so vzbujena vezana stanja dveh atmov, podobna molekuli, ki
disociirajo v osnovnem stanju. Za laserje so zanimivi predvsem ekscimerji
te"zkih "zlahtnih plinov (Xe, Kr, Ar) in halogenov (F, Cl, Br, I), ker jih
je mogo"ce dokaj u"cinkovito pridobivati in ker dajejo lasersko svetlobo v
ultravijoli"cnem podro"cju med 200~nm in 400~nm, ki ga drugi laserski
sistemi le te"zko pokrivajo.

Vezano stanje dveh atomov dobimo, kadar je ionizacijska energija prvega
atoma manj"sa od vsote elektronske afinitete drugega atoma in
elektrostati"cne energije vezave obeh ionov. Vzemimo za primer klor in
kripton. Ionizacijska energija kriptona v osnovnem stanju je 14~eV, v
vzbujenem pa 5~eV. Elektronska afiniteta klora je 3.75~eV in
elektrostati"cna vezavna energija KrCl okoli 7~eV. Tako je za formiranje
molekule KrCl v osnovnem stanju potrebno dodati okoli 4~eV, pri tvorbi
molekule v vzbujenem stanju pa se sprosti okoli 6~eV. Pribli"zno obliko
celotne potencialne energije molekule KrCl v osnovnem in vzbujenem stanju
ka"ze slika \ref{s6.6}. Molekula, ki je vezana v vzbujenem stanju, po
sevalnem prehodu v osnovno stanje takoj razpade, zato je zelo lahko dose"ci
obrnjeno zasedenost.

Razpadni "cas vezanega stanja je blizu 10~ns, spodnjega nevezanega pa okoli
0.1~ps. Zato je spektralna "sirina prehoda precej velika, okoli 1~nm, in
laser, ki ta prehod uporablja, lahko deluje v ve"cjem delu tega intervala.

Ekscimeri se formirajo v me"sanici halogenega in "zlahtnega plina, ki ga
vzbujamo z m"cnim elektronskim snopom. Ekscimerni laserji delujejo
ve"cinoma v sunkih s precej veliko energijo, do 100~kJ z dodatnimi
oja"cevalnimi stopnjami. Imajo tudi dober izkoristek, do 8\%, kar vse
pojasnjuje veliko zanimanje za ekscimerne laserje.

\section{Laserji na organska barvila}

Posebej zanimivi so sevalni prehodi z veliko spektralno "sirino. V
frekven"cnem intervalu takega prehoda je mo"zno spreminjati frekvenco
laserja, kar je posebej za uporabo v spektroskopiji izrednega pomena. Eno
mo"znost nudijo organska barvila.

Pribli"zno shemo energijskih nivojev molekule tipi"cnega organskega barvila,
znan primer je rodamin 6G, prikazuje slika \ref{6.7}. Vsa elektronska stanja
so razcepljena v vibracijska in rotacijska podstanja. Rotacijska stanja so
tako blizu skupaj, da se v raztopini zaradi trkov zlijejo med seboj v zvezen
pas, ki ima tipi"cno "sirino do 50~nm. Elektronska stanja so lahko singletna
(S), ki imajo elektronski spin 0, in tripletna (T) z elektronskim spinom 1.
Elektri"cni dipolni prehodi med tripletnimi in singletnimi stanji so
prepovedani, zato je najni"zje tripletno stanje metastabilno.

V toplotnem ravnovesju je molekula na dnu osnovnega elektronskega stanja S$_0
$. Z absorpcijo vidne svetlobe primerne frekvence preide v nekam v vzbujeno
stanje $S_1$. Preko trkov z molekulami topila vzbujena barvilna molekula
zelo hitro, v "casu okoli pikosekunde preide na dno vzbujenega stanja, od
koder s sevanjem preide nekam v osnovno stanje $S_0$, od koder zopet s trki
hitro preide nazaj na dno osnovnega stanja, kot ka"ze slika \ref{s6.7}. Ker
sta obe elektronski stanji raz"sirjeni zaradi vibracijskih in rotacijskih
stanj, sta absorpcijska in emisijska fluorescen"cna "crta "siroki. Tipi"cna
"sirina je blizu 50~nm. Energija izsevane svetlobe je zmanj"sana za energijo
prehodov s trki, zato je fluorescen"cna "crta premaknjena k ni"zjim
frekvencam od absorpcijske. Absoprcijski in fluoresecen"cni spekter prehoda $%
S_0-S_1$ ka"ze slika \ref{s6.8}.

Absorpcijski presek med osnovnim in vzbujenim singletnim stanjem je velik,
sevalni razpadni "cas z dna stanja $S_1$ pa dolg v primeri z nesevalnimi
prehodi z vibracijsko- rotacijskih nivojev stanja $S_0$ na njegovo dno, zato
je lahko dobiti oja"cenje v fluorescen"cni "crti. Barvilni laser "crpamo s
svetlobo z nekaj vi"sjo frekvenco, ki ustreza vrhu absorpcijske "crte. Pri
tem ve"cina molekul, ki so se vzbudile z absoprcijo, sodeluje pri
stimulirani emisiji, zato je izkoristek pri pretvorbi "crpalne svetlobe v
mo"c laserja lahko zelo velik.

Energija tripletnega stanja $T_1$ se deloma prekriva s stanjem $S_1$, zato
so mo"zni prehodi s trki iz $S_1$ v $T_1$. Ker je tripletno stanje
metastabilno, se lahko v njem nabere znatno "stevilo molekul barvila. Zaradi
tega se zmanj"sa "stevilo molekul v singletnem stanju, poleg tega pa je
mo"zna absorpcija iz stanja $T_1$ v $T_3$, kar lahko prepre"ci lasersko
delovanje med stanjema $S_1$ in $S_0$. Tej te"zavi se izognemo, "ce laser
deluje le v sunkih, pri stacionarnem delovanju pa tako, da raztopina barvila
kro"zi skozi laser.

Barvilni laser lahko deluje pri vseh frekvencah znotraj "siroke
fluorescen"cne "crte. Zato moramo v resonator vgraditi nek frekven"cno
selektiven element, s katerim lahko nastavljamo frekvenco laserja. Uporabna
je prizma kot v primeru Ar laserja ali kombinacije interferometrov. Zanimiva
mo"znost je, da eno od zrcal nadomestimo z uklonsko mre"zico, ki je
postavljena pod takim kotom, da se po osi resonatorja odbije svetloba
"zeljene valovne dol"zine (slika \ref{s6.9}. To lahko spremenimo s
spreminjanjem kota nagiba mre"zice.

Barvilne lahko laserje "crpamo z bliskovno lu"cjo. Danes pogosteje uporabimo
drug laserjem primerne valovne dol"zine, na primer Ar ali ekscimerni laser.
Pri tem zaradi dobrega izkoristka barvila ne izgubimo mnogo mo"ci, pridobimo
pa mo"znost nastavljanja frekvence. Z menjavo barvil tako zvezno pokrijemo
ves vidni del spektra.

"Siroko obmo"cje oja"cevanja barvila nam z uklepanjem faz omogo"ca dobiti
tudi zelo kratke svetlobne sunke. V prej"snjem poglavju smo videli, da je
dol"zina sunka iz fazno uklenjenega laserja obratno sorazmerna s spektralno
"sirino oja"cevalne "crte. Iz barvilnega laserja zato lahko dobimo zelo
kratke sunke, pod 1~ps.

\section{Titan-safirni laser}

Podobne lastnosti kot barvila imajo nekateri prehodni elementi, vgrajeni v
primerno kristalno mre"zo. Posebej velja omeniti safir, to je kristal Al$_2$O%
$_3$, s primesjo titana. Prehod Ti v bli"znjem infrarde"cem obmo"cju je
zaradi interakcije s fononi mo"cno raz"sirjen, tako da je mo"zno dobiti
oja"cevanje v obmo"cju od 700 do 1000~nm. V tem intervalu je titan-safirnemu
laserju mogo"ce nastavljati frekvenco. "Crpamo ga navadno z drugim laserjem,
najve"ckrat z argonskim.

Titan-safirni laser zelo dobro deluje v sunkih z uklepanjem faz. Zaradi
velike spektralne "sirine oja"cevanja so sunki izredno kratki, okoli 100~ps.

\section{Polvodni"ski laserji}

Za "siroko uporabo so danes brez dvoma najpomembnej"si polvodni"ski laserji.
Njihove glavne zna"cilnosti so majhne dimenzije, neposredno "crpanje z
elektri"cnim tokom majhne jakosti in pri nizki napetosti, velik izkoristek,
preko 20%,
pri pretvorbi elektri"cne mo"ci v svetlobno, in mo"znost hitre modulacije
svetlobne mo"ci z modulacijo elektri"cnega toka. Poleg tega jih je mo"c
integrirati z drugimi polvodni"skimi elementi in se za njihovo izdelavo
uporablja obi"cajna polvodni"ska tehnologija.

Polvodniki absorbirajo svetlobo s frekvenco nad energijo prehoda iz
valen"cnega v prevodni pas. Pri tem z vzbuditvijo elektrona iz valen"cnega v
prevodni pas nastane par elektron-vrzel. Ta se lahko rekombinira z
izsevanjem fotona, kar ustreza spontanemu ali stimuliranemu prehodu v atomih
in molekulah. Zaradi ohranitve gibalne koli"cine se valovni vektor elektrona
pri prehodu le zelo malo spremeni, saj je valovni vektor vidne svetlobe
majhen v primerjavi z vektorjem recipro"cne mre"ze polvodni"skega kristala.
To ima pomembno posledico. Najobi"cajnej"sa polvodnika, silicij in germanij,
imata vrh prevodnega pasu v centru Brilluinove cone, to je pri valovnem
vektorju ni"c, dno prevodnega pasu pa pri kon"cno velikem valovnem vektorju.
Zato direkten prehod z dna prevodnega pasu v vrh valen"cnega pasu z
izsevanjem fotona ni mo"zen, potrebna je so"casna emisija ali absorpcija
fonona, ki poskrbi za ohranitev gibalne koli"cine. Tak proces pa je seveda
mnogo manj verjeten, zato v siliciju in germaniju v obi"cajni obliki ni
mogo"ce dobiti znatnega sevanja s prehodi iz prevodnega v valen"cni pas.

Opisane te"zave ni pri spojinah iz tretje in pete skupine elementov, na
primer GaAs, kjer je valen"cni pas 1,4 eV nad prevodnim. Na njih so osnovani
polvodni"ski laserji.

Verjetnost za zasedenost elektronskih stanj v termi"cnem ravnovesju v
polvodniku je podana s Fermi-Diracovo funkcijo \ref{ts}:

\begin{equation}  \label{6.1}
f(E)=\frac{1}{e^{(E-E_F)/k_B T}+1}\;\;.
\end{equation}
Pri nizkih temperaturah so zasedena vsa stanja do Fermijeve energije $E_F$,
nad njo pa so prazna. V "cistem polvodniku je $E_F$ na sredini med
valen"cnim in prevodnim pasom, z dodajanjem donorskih primesi pa se
pribli"zuje prevodnemu pasu. "Ce je koncentracija primesi dovolj velika,
lahko $E_F$ tudi pri absolutni ni"cli pride nad dno prevodnega pasu. Tak
degenriran polvodnik se obna"sa podobno kot kovina.

"Ce polvodnik ni v termi"cnem ravnovesju, na primer v p-n spoju, skozi
katerega te"ce elektri"cni tok, lahko vpeljemo pribli"zni Fermijevi energiji 
$E_{Fv}$ in $E_{Fp}$ posebej za valen"cni in prevodni pas. Tak pribli"zek je
dober, kadar je "cas, v katerem preidejo vzbujeni elektroni z vi"sjih
energij na dno prevodnega pasu, dosti kraj"si od "casa prehoda iz prevodnega
v valen"cni pas. Ta pogoj je v polvodnikih, ki jih uporabljamo za laserje,
navadno zelo dobro izpoljnen. Preko trkov s fononi elektroni preidejo v dno
prevodenga pasu v pikosekundah, za prehod v valen"cni pas, to je za
rekombinacijo elektrona z vrzeljo, pa so potrebne nanosekunde.

Neravnovesno stanje, ki ga lahko opi"semo z pribli"znima Fermijevima
energijama, dobimo, kadar v degeneriran polvodnik p-tipa z dovolj veliko
hitrostjo dodajamo elektrone v prevodni pas. To lahko storimo preko p-n
spoja, na katerem je napetost v prevodni smeri. Tedaj dobimo polo"zaj, ki ga
ka"ze slika \ref{s6.10}. Naj bo $N_p$ gostota elektronov v prevodnem pasu, $I
$ elektri"cni tok skozi spoj, $\tau$ pa "cas za rekombinacijo elektrona in
vrzeli. Velja 
\begin{equation}  \label{6.2}
\frac{N_p}{\tau}=\frac{I}{e V}\;,
\end{equation}
kjer je V volumen, v katerem se elektroni nahajajo. Po drugi strani je $N_p$
podana kot integral gostote elektronskih stanj v prevodnem pasu $%
\rho_p(E)=1/(2\pi^2)(2m_p/\hbar^2)^{3/2}(E-E_g)^{1/2}$, kjer je $m_p$
efektivna masa prevodnih elektronov, in Fermi- Diracove funkcije: 
\begin{equation}  \label{6.3}
N_p=\int_{0}^{\infty}\,\rho_p(E)f_p(E)\,dE
\end{equation}
Iz ena"cb \ref{6.2} in \ref{6.3} lahko izra"cunamo vrednost $E_{Fp}$ pri
poljubni temperaturi. Pri $T=0$ je integral preprost in je 
\begin{equation}  \label{6.4}
E_{Fp}(T=0)=(2\pi^2)^{2/3}\frac{\hbar^2}{2m_p}\,N_p^{2/3}
\end{equation}
$E_{Fp}$ je odvisna od toka skozi spoj in od temperature, kar je pomebno za
delovanje polvodni"skega laserja.

Naj na p-n spoj vpada svetlobni snop s frekvenco $\omega$, ki je nekoliko
nad $E_g/\hbar$. Svetloba povzro"ca prehode med stanji z energijo $E_p$ v
prevodnem pasu in med stanji z energijo $E_v$ v valen"cnem pasu, kot ka"ze
slika \ref{s6.11}. Ta stanja se lahko razlikujejo po valovnem vektorju, zato
moramo verjetnost za stimuliran prehod najprej zapisati za dolo"cen valovni
vektor $\vec{k}$, nato pa se"steti po vseh mo"znih $\vec{k}$. V primeru
izotropnih elektronskih pasov je verjetnost za prehod odvisna le od
velikosti $\vec{k}$. Uporabimo zlato pravilo, kjer upo"stevamo, da je
verjetnost za zasedenost gornjega stanja $f_p(E_p)$ in verjetnost, da je
spodnje stanje nezasedeno, $1-f_v(E_v)$: 
\begin{equation}  \label{6.5}
w_s(k)=\frac{2\pi}{\hbar}|H_{pv}|^2\delta(E_p-E_v- \hbar\omega)
f_p(E_p)[1-f_v(E_v)]\;,
\end{equation}
kjer je $H_{pv}= |p\rangle \hat{x}|v\rangle E $ matri"cni element za dipolni
prehod v svetlobnem polju $E$ med prevodnim in valen"cnim pasom. Podobno je
verjetnost za absorpcijo 
\begin{equation}  \label{6.6}
w_a(k)=\frac{2\pi}{\hbar}|H_{pv}|^2\delta(E_p-E_v- \hbar\omega)
f_v(E_v)[1-f_p(E_p)]\;.
\end{equation}

Razliko med "stevilom spontanih emisij in absorpcij na enoto volumna dobimo,
"ce razliko gornjih verjetnosti se"stejemo po vseh $k$: 
\begin{eqnarray}  \label{6.7}
N_{pv}-N_{vp}&=&\int[w_s-w_a]\rho(k)\,dk  \nonumber \\
&=&\frac{2}{\pi\hbar}|H_{pv}|^2 \int[f_p(E_p)-f_v(E_v)]
\delta(E_p-E_v-\hbar\omega) k^2\,dk\;.
\end{eqnarray}
Upo"stevli smo, da je gostota stanj $\rho(k)=1/\pi^2 k^2dk$. "Stejmo
energijo od vrha valen"cnega pasu. Energija elektronov blizu dna prevodnega
pasu je $E_g+1/(2m_e)\hbar^2 k^2$, vrzeli pri vrhu prevodnega pasu pa $%
1/(2m_h)\hbar^2 k^2$, kjer sta $m_e$ in $m_h$ efektivni masi elektronov in
vrzeli. Tako je 
\begin{equation}  \label{6.8}
E_p-E_v=\hbar\omega^{\prime}=\frac{\hbar^2 k^2}{2}(\frac{1}{m_h}+ \frac{1}{%
m_e})+E_g=\frac{hbar^2 k^2}{2m_r}+E_g\;,
\end{equation}
kjer smo z $m_r=m_e m_h/(m_e+m_h)$ ozna"cili reducirano maso lektrona in
vrzeli. Iz zadnjega izraza dobimo 
\begin{equation}  \label{6.9}
k^2=\frac{2m_r}{\hbar}(\omega^{\prime}-\frac{E_g}{\hbar})
\end{equation}
in 
\begin{equation}  \label{6.10}
k^2dk=\frac{1}{2}(\frac{2m_r}{\hbar})^{3/2}\sqrt{\omega^{\prime}- \frac{E_g}{%
\hbar})}\;.
\end{equation}
V \ref{6.7} preidemo z integracije po $k$ na $\omega^{\prime}$ in dobimo z
upo"stevanjem lastnosti delta-funkcije 
\begin{equation}  \label{6.11}
N_p-N_v=\frac{1}{\pi\hbar^2}(\frac{2m_r}{\hbar})^{3/2} \sqrt{\omega-\frac{E_g%
}{\hbar}}[f_p(E_p)-f_v(E_v)]\;.
\end{equation}
Zaradi \ref{6.9} je "se $E_p=m_r/m_e(\omega-E_g/\hbar)+E_g$ in $%
E_v=m_r/m_h(\omega-E_g/\hbar)$.

Dobljeni izraz je sorazmeren z oja"cenjem vpadne svetlobe v polvodniku.
Oja"cenje bo o"citno realno le, "ce bo $\omega$ ve"cja od $E_g/\hbar$. Pri
ni"zjih frekvencah seveda sploh ni prehodov in polvodnik je prozoren. Da
bomo imeli res oja"cenje in ne absorpcije, mora biti tudi $f_p(E_p)>f_v(E_v)$%
, torej 
\begin{equation}  \label{6.12}
\frac{1}{e^{(E_p-E_{Fp})}+1}>\frac{1}{e^{(E_v-E_{Fv})}+1}\;.
\end{equation}
Gornja neena"cba bo veljala, "ce bo 
\begin{equation}  \label{6.13}
\hbar\omega<E_{Fp}-E_{Fv}\;.
\end{equation}
To je osnovni pogoj za oja"cevanje v polvodnikih. Oja"cujejo se lahko tiste
frekvence, ki so nad energijsko "spranjo $E_g$ in pod razliko pribli"znih
neravnovesnih Fermijevih energij za oba pasova. Oja"cenje kot funkcijo
frekvence ka"ze slika \ref{s6.12}.

Vrednosti $E_{Fp}$ in $E_{Fv}$ sta odvisni tako od toka skozi p-n spoj kot
od temperature. Z njima je povezan tudi vrh oja"cenja, kar je mogo"ce
izkoristiti za spreminjanje frekvence polvodni"skega laserja s tokom ali
temperaturo.

Oja"cenja v polvodni"skih laserjih je precej veliko, lahko ve"c od 100~cm$%
^{-1}$. Zato je mogo"ce dobiti delujo"c laser "ze v zelo kratkem aktivnem
volumnu le nekaj mikronov. Obi"cajni polvodni"ski laserji so dolgi okoli
0.25~mm.

Najpomembnej"si polvodni"si laserji so osnovani na na sistemu GaAs z Ga$%
_{1-x}$Al$_{x}$As in na sistemu Ga$_{1- x}$In$_{x}$As$_{1-y}$P$_y$. Prvi
delujejo od 750 nm do 880 nm, odvisno od $x$ in koncentracije primesi, drugi
pa med 1,1~$\mu$m in 1,6~$\mu$m in so posebno pomembni za opti"cne
komunikacije, ki najve"krat delujejo pri 1,3~$\mu$m in 1,55~$\mu$m. Poglejmo
si delovanje galij-arsenidnega laserja nekoliko pobli"ze.

Galij-arsenidni laser je narejen iz plasti GaAs in Ga$_{1- x}$Al$_x$As z
ustreznimi primesmi, kot ka"ze slika \ref{s6.13}. Take plastne strukture je
naredijo z epitaksialno rastjo. Pri tem je pomembno, da so medatomske
razdalje plasti "cim bolj enake, da na meji dveh plasti ni napak. Ga$_{1-x}$%
Al$_x$As ima energijsko "spranjo med vvalen"cnim in prevodnim pasom nekoliko
ve"cjo kot "cisti GaAs, poleg tega ima tudi nekoliko ve"cji lomni koli"cni.
Obe lastnosti sta pomembni za delovanje laserja. Aktivna plast je tanka,
okoli 0,2~$\mu$m debela plast "cistega GaAs. Potek energije pasov preko
aktivne plasti z napetostjo v prevodni smeri ka"ze slika \ref{s6.14}.
Elektroni te"cejo iz n-tipa Ga$_{1-x}$Al$_x$As v prevodni pas aktivne plasti
GaAs, vrzeli pa iz p Ga$_{1-x}$Al$_x$As v valen"cni pas. Zaradi manj"se
energijske "spranje so tako elektroni kot vrzeli ujeti, da ne morejo
difundirati iz aktivne plasti in je zato koncentracija elektronov v
prevodnem pasu in vrzeli v valen"cnem "ze pri razmeroma majhnih tokovih
lahko velika in je izpolnjen pogoj za oja"cevanje svetlobe.

Zaradi manj"sega lomnega koli"cnika GaAs od Ga$_{1- x}$Al$_x$As je laserska
svetloba ujeta v aktivni plasti, podobno kot v opti"cnih vlaknih, kot ka"ze
slika \ref{s6.15}. To dodatno zmanj"suje izgube, ker prepre"cuje absorpcijo
v podro"cju izven aktivne plasti, kjer ni izpolnjen pogoj za oja"cevanje.

Resonatorska zrcala so navadno kar gladko odklane stranske ploskve
polvodni"skega kristala, ki imajo zaradi velikega lomnega koli"cnika ($n=3,6$%
) dovolj veliko reflektivnost za u"cinkovito delovanje laserja.

Pri strukturi, ki jo ka"ze slika \ref{s6.13}, je aktivna plast v pre"cni
smeri neomejena, zato lahko hkrati sveti mnogo pre"cnih nihanj, zaradi
"cesar je slab"sa pre"cna koherenca snopa in delovanje laserja nestabilno.
To slabost popravijo tako, da plasti ob straneh pojedkajo, da ostane le
kakih 10~$\mu$m "sirok greben, kot ka"ze slika \ref{s6.16}. Odjedkani
material nadomestijo s "cistim Ga$_{1-x}$Al$_x$As, tako da je aktivni
volumen od vseh strani obdan s snovjo z ve"cjim lomnim koli"cnikom. S tem
dobimo pravokoten svetlobni vodnik, v katerem je ujet laserski snop. Ob
primerni izbiri dele"za aluminija dobijo take lomne koli"cnike, da je v
laserju mo"zen le osnovni snop brez vozlov v pre"cni smeri. Izhodni snop iz
laserja je seveda elipti"cen s presekom okoli 1~$\mu$m v navpi"cni in 10~$\mu
$m v pre"cni smeri. To da v ve"cji oddaljenosti snop z divergenco kakih 70$^o
$ v navpi"cni in okoli 5$^o$ v pre"cni smeri. "Ce potrebujemo cilindri"cno
simetri"cen snop, ga moramo popraviti z ustreznimi cilindri"cnimi le"cami
(Naloga).

Galij-arsenidni laser, kakr"sen je prikazan na sliki \ref{s6.16} lahko
deluje "ze pri "crpalnem toku nekaj miliamperov. Tipi"cni tokovi so med
50~mA in 100~mA. Ker se velik dele"z elektonov in vrzeli rekombinira s
sevanjem v aktivni plasti, je izkoristek GaAs laserjev velik, tudi preko 30%
%. Tipi"cna izhodna mo"c je tako reda velikosti
10~mW. Zaradi velike gibljivosti elektronov in vrzeli v GaAs je mogo"ce tok
in s tem izhodno svetlobno mo"c tudi zelo hitro modulirati, do nekaj GHz,
kar je pomembno za uporabo v opti"cnih komunikacijah.
