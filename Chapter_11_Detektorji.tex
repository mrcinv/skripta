\chapterimage{CCD.jpg} % Chapter heading image

\chapter{Detektorji svetlobe}

V tem poglavju bomo spoznali detektorje svetlobe, ki so nepogrešljivi
pri kvantitativni obravnavi optičnih pojavov. Detektorji se med seboj razlikujejo
po načinu delovanja in po svojih specifikacijah, ki jih bomo opisali v nadaljevanju. Največ
pozornosti bomo posvetili polprevodniškim detektorjem, ki so danes najbolj razširjeni.
Na koncu bomo spoznali še šum pri detekciji, ki omejuje uporabnost naprav.

\section{Osnovne karakteristike detektorjev}

Osnovna naloga optičnih detektorjev je pretvoriti vpadni svetlobni signal 
v nek drug signal, ki ga lahko natančno merimo. Navadno sta to električni tok 
ali električna napetost, ki sta sorazmerna z močjo vpadne svetlobe 
in ne z amplitudo električne poljske jakosti. V grobem delimo detektorje v dve skupini, 
na termične in kvantne. Prvi pretvorijo energijo vpadne svetlobe 
v toploto, drugi pa temeljijo na fotoefektu, kjer vpadli foton izbije elektron ali 
ustvari par elektron-vrzel.

Pri termičnih detektorjih zaznamo svetlobo tako,
da merimo povečanje temperature senzorja zaradi absorbirane svetlobe in taki detektorji
zaznavajo energijo vpadle svetlobe. Njihov odziv je razmeroma počasen, zato jih uporabljamo
predvsem za merjenje optične moči, lahko tudi zelo velike. Po drugi strani pa je odziv
termičnih detektorjev neodvisen
od valovne dolžine vpadne svetlobe, zaradi česar so termični detektorji uporabni na 
širokem območju od globoke ultravijolične do daljne infrardeče svetlobe. Uporaba
prevlada predvsem v infrardrečem, teraherčnem ali celo mikrovalovnem območju, kjer so 
drugi detektorji bistveno manj občutljivi. 
Primeri termičnih detektorjev so bolometer, termočlen in piroelektrični detektor.

Druga skupina so kvantni detektorji, v katerih se
fotoni absorbirajo in povzročijo pojav prostih nosilcev naboja. Taki detektorji
zaznavajo število vpadlih fotonov. Odlikuje jih zelo hiter odziv 
(tipično pod $\si{\micro\second}$)
in velika občutljivost. Njihova poglavitna slabost je omejen obseg valovnih dolžin,
pri katerih zaznavajo svetlobo, poleg tega jih je za optimalno delovanje treba 
hladiti. Primeri so vakuumske, polprevodniške in plazovne fotodiode.
\begin{figure}[h]
\centering
\def\svgwidth{65truemm} 
\input{slike/11_ShemaTermKv.pdf_tex}
\caption{Primerjava spektralnega odziva termičnega in kvantnega detektorja}
\label{fig:shemaTermKv}
\end{figure}

Osnovne karakteristike, ki omogočajo primerjavo med detektorji in določajo njihovo uporabnost,
so občutljivost, spektralni odziv, odzivni čas in prag detekcije. 

\begin{enumerate}
\item Občutljivost detektorja $R$ pove, koliko je izhodnega signala 
na enoto vpadnega svetlobnega toka. Enota za občutljivost je tako A/W ali V/W. 
\item Spektralni odziv pove, kako se občutljivost spreminja z valovno dolžino $R(\lambda)$.
Pri termičnih detektorjih je $R(\lambda)$ konstanta, medtem ko kvantni detektorji 
delujejo le v določenem območju valovnih dolžin, ki je odvisen od snovi, 
iz katere je detektor narejen. 
\item Odzivni čas pove, kako hitro se detektor odzove na spremembo optičnega signala. Predvsem 
optične telekomunikacije zahtevajo izredno hiter odziv.
\item Prag detekcije pove, pri kolikšni vpadni svetlobni moči postane razmerje med signalom ($S$)
in šumom ($N$, {\it noise}) enako $S/N = 1$. 
\end{enumerate}

\section{Termični detektorji}
Termične detektorje se zaradi njihovega razmeroma počasnega odziva uporablja predvsem 
za merjenje vpadne moči in za detekcijo svetlobe tistih valovnih dolžin, za katere 
ni drugih preprostih ali učinkovitih detektorjev. Pogosto se uporabljajo za termografske 
kamere in v astronomiji.

Delovanje termičnih detektorjev temelji na spremembi temperature zaradi absorpcije svetlobe 
(energije), detektorji pa se med seboj razlikujejo predvsem v načinu pretvorbe spremembe 
temperature v električni signal.
Tipalo termičnih detektorjev mora biti pri vseh vrstah dobro počrnjeno, da absorbira
svetlobo v čim širšem spektralnem območju. Čeprav je njihova občutljivost načeloma 
neodvisna od valovne dolžine vpadne svetlobe, se v praksi pojavijo omejitve zaradi
prepustnosti okna in absorpcijskega spektra črnega nanosa. Tipala so majhna, zato 
da dosežemo čim hitrejši odziv, ki pa je kljub temu navadno počasnejši od 1~ms. Sodobnejši
detektorji se po odzivnem času že približujejo kvantnim, saj dosegajo odzivne čase tudi do
$\sim 10~\si{\micro\second}$. Termične detektorje uporabljamo pri sobni temperaturi, 
za zahtevne meritve pa jih hladimo na nekaj K. 

Obravnavajmo termični detektor, katerega tipalo naj ima toplotno kapaciteto $C$. Toplota
se s tipala odvaja v nek toplotni zalogovnik s temperaturo $T_0$, 
toplotne izgube pa označimo z $\Lambda$. Ko na tipalo vpada svetloba moč $P$, začne
temperatura tipala $T$ zaradi absorpcije svetlobe naraščati, hkrati pa se tipalo 
ohlaja zaradi odtekanja toplote:
\beq
\frac{dW}{dt} = C \frac{dT}{dt} = P - \Lambda (T-T_0).
\label{TD1}
\eeq
V stacionarnem stanju, ki ga dosežemo pri konstantnem vpadnem svetlobnem toku, se
temperatura tipala ne spreminja in razlika temperature tipala in zalogovnika je 
\beq
T - T_0 = \frac{P}{\Lambda}.
\label{temp_sens}
\eeq
Občutljivost detektorja, ki je sorazmerna z razliko temperatur, 
je torej obratno sorazmerna s toplotnimi izgubami. Za večjo občutljivost moramo
torej toplotne izgube detektorja kar se da zmanjšati. 

Po enačbi~(\ref{TD1}) se temperatura približuje stacionarni vrednosti s časovno konstanto 
\beq
\tau = \frac{C}{\Lambda},
\label{TermD_t}
\eeq
ki je ključni parameter za določanje odzivnega časa detektorja. Odzivni
čas je sorazmeren s kapaciteto senzorja, zato so tipala praviloma zelo majhna.
Vidimo, da moramo za dosego čim krajšega odzivnega časa toplotne izgube kar se da povečati. Če velike
izgube skrajšajo odzivni čas, pa po drugi strani zmanjšajo občutljivost (enačba~\ref{temp_sens}), 
zato pri termičnih detektorjih ne moremo imeti hkrati velikega in hitrega odziva. 
Če želimo toplotne izgube povečati, da s tem skrajšamo odzivni čas, detektorje hladimo z zrakom 
ali celo z vodo, majhne toplotne izgube pa so omejene s sevanjem.  

Podrobneje poglejmo odziv termičnega detektorja od vpadne moči. Naj se vpadna moč
spreminja s časom, temperatura na detektorju pa temu sledi z določeno zakasnitvijo. Odziv
najlepše izračunamo v Fourierevem prostoru. Vpadno moč in temperaturo izrazimo kot
\beq
P(t) = \int_{-\infty}^{\infty} P_\omega e^{i\omega t}d\omega \quad \mathrm{in} \quad
T = T_0 + \int_{-\infty}^{\infty} T_\omega e^{i\omega t}d\omega.
\label{TermTF}
\eeq
To vstavimo v enačbo~(\ref{TD1}) in dobimo
\beq
\int_{-\infty}^{\infty} i \omega T_\omega e^{i\omega t}d\omega = \frac{1}{C}
\int_{-\infty}^{\infty} (P_\omega - \Lambda T_\omega) e^{i\omega t}d\omega.
\eeq
Enačbi zadostimo, če izenačimo člene pred vsako spektralno komponento posebej
\beq
i \omega T_\omega = \frac{1}{C}\left(P_\omega - \Lambda T_\omega\right).
\eeq
Če vpeljemo odzivni čas $\tau$ (enačba~\ref{TermD_t}), sledi
\beq
T_\omega = \frac{1}{\Lambda}\left(\frac{1}{1+i \omega \tau}\right)P_\omega.
\label{TermOdziv}
\eeq

\begin{definition}
Pokaži, da je odziv termičnega detektorja na zelo kratek svetlobni sunek oblike 
$P(t) = P_0 \delta(t-t_0)$ enak 
\beq
T(t)=\frac{iP_0}{\Lambda}e^{-(t-t_0)/\tau}.
\eeq
\end{definition}

\subsection*{Bolometer}
Bolometer je termični detektor, pri katerem zaznavamo spremembo električne upornosti
zaradi spremembe temperature tipala\footnote{Prvi bolometer je leta 1881 naredil
ameriški fizik, astronom in letalski inženir Samuel Pierpont Langley, 1834--1906.}. 
Tipalo je praviloma počrnjena tanka ploščica, 
navadno je narejena iz termistorja, polprevodnika ali superprevodnika. Tipalo preko
referenčnega upora priključimo na napetost, preko kondenzatorja pa merimo napetost na njem.
Za meritve konstantega svetlobnega toka tipalo navadno vežemo v Wheatstonov mostiček. V obeh
primerih za referenčni upor vzamemo kar enako tipalo, ki ga zaščitimo pred vpadno svetlobo, 
tako da postane sistem neobčutljiv na morebitne spremembe temperature okolice.
\begin{figure}[h!]
\centering
\def\svgwidth{85truemm} 
\input{slike/11_bolometer.pdf_tex}
\caption{Shema bolometra}
\label{fig:Bolometer-shema}
\end{figure}

Prednost bolometrov s termistorjem je približno linearna zveza med upornostjo in 
temperaturo. Uporabljamo jih predvsem za merjenje večjih vpadnih moči, saj taki 
detektorji niso zelo občutljivi ($R\sim 100~\si{\volt/\watt}$). 
So pa robustni, stabilni in delujejo pri sobni 
temperaturi. Odzivni časi so okoli $\tau \sim 1$--$20~\si{\milli\second}$. 
Pri polprevodniških bolometrih upornost pojema eksponentno s temperaturo. 
Primerni so za detekcijo teraherčnih valovanj, vendar mora biti za ta namen 
bolometer (npr. germanijev) hlajen s tekočim helijem. Tako lahko dosežemo
občutljivosti večje od $R \sim 10^8~\si{\volt/\watt}$. Zelo občutljivi so tudi 
detektorji s superprevodnimi tipali, saj je odvisnost upornosti od temperature v bližini
prehoda v superprevodno stanje zelo velika ($R \sim 10^3~\si{\volt/\watt}$).

\begin{figure}[h]
\centering
\includegraphics[width=80truemm]{slike/11_Bolometer.jpg}
\caption{Bolometer za merjenje prasevanja. Premer kovanca za primerjavo je 18~mm. 
Vir: NASA/JPL-Caltech.}
\label{fig:Bolometer}
\end{figure}

\subsection*{Termočlen}
Termočlen je sestavljen iz dveh različnih vodnikov. En spoj vodnikov počrnimo, drugega, 
referenčnega, pa zaščitimo pred svetlobo. Zaradi vpadne svetlobe se počrnjeni spoj 
segreje, med obema spojema nastane temperaturna razlika in zaradi termoelektričnega 
pojava tudi električna napetost, ki jo lahko merimo. Pri tem pazimo, da je prevodnost
vodnikov čim večja, toplotna prevodnost pa čim manjša. Odzivni čas termočlenov je 
okoli $\tau \sim 10$--$20~\si{\milli\second}$, občutljivost pa okoli $R \sim 10~\si{\volt/\watt}$.
Ker so napetosti, ki se pojavijo med stikoma, razmeroma majhne (le okoli 
$\sim 10~\si{\micro\volt/K}$) pogosto vežemo več (nekaj deset) termočlenov zaporedno v
termobaterijo. Občutljivost s tem naraste na $R \sim 200~\si{\volt/\watt}$, podaljša 
pa se časovna konstanta $\tau \sim 10$--$2000~\si{\milli\second}$. Prednost termočlenov je,
da za svoje delovanje ne potrebujejo zunanjega napajanja. 

\subsection*{Piroelektrični detektor}
Piroelektriki so snovi brez centra inverzije, v katerih je lastna električna 
polarizacija odvisna od temperature (npr. LiTaO$_3$, triglicin sulfat TGS 
in vsi feroelektriki). Piroelektrični detektor je narejen iz 
ploščice piroelektrične snovi med dvema elektrodama oziroma ploščama kondenzatorja.
Ko se ploščica zaradi absorbirane svetlobe segreje, se ji spremeni polarizacija in 
med elektrodama se pojavi premikalni tok, ki ga merimo na merilnem uporniku.

Zveza med spremembo temperature in spremembo polarizacije je
\beq
dP = a dT,
\eeq
kjer je $a$ piroelektrični koeficient. 

Med obema elektrodama s površino $S$ preteče naboj
\beq
de = I dt = S dP = S a dT.
\eeq
Tok skozi tipalo je tako
\beq
I = S a \frac{dT}{dt}.
\label{piro}
\eeq
Piroelektrični detektor je torej občutljiv na časovni odvod temperature detektorja, 
s tem pa tudi na spreminjanje vpadne svetlobne moči. V stacionarnem stanju 
detektor ne proizvaja električnega toka, zato moramo za merjenje 
konstantnega svetlobnega toka vpadno svetlobo najprej modulirati.
Navadno to naredimo kar z mehanskim zaklopom. Piroelektrični detektorji
se večinoma uporabljajo kot preprosti infrardeči detektorji. 
Njihova občutljivost je $R \sim 1~\si{\micro\ampere/\watt}$, odzivni čas pa odvisen od 
upornika v vezju, ampak lahko doseže vrednosti $\tau \sim 10~\si{\micro\second}$.

Poglejmo temperaturni odziv na tipalu. Izhajamo iz enačb~(\ref{TermTF}), (\ref{TermOdziv}) in
(\ref{piro}) in izračunajmo tok $I$ v odvisnosti od frekvence modulacije.
\beq
I = Sa \frac{dT}{dt} = Sa \frac{d}{dt} \int_{-\infty}^{\infty} T_\omega e^{i\omega t}d\omega 
=Sa\int_{-\infty}^{\infty}\frac{1}{\Lambda}\left(\frac{P_\omega}{1+i \omega \tau}\right) \,i \omega\,
e^{i\omega t}d\omega.
\eeq
Sledi 
\beq
I_\omega = \frac{i \omega\, SaP_\omega/\Lambda}{1 + i \omega \tau}.
\eeq
Vidimo, da pri majhnih frekvencah tok narašča, pri velikih frekvencah pa postane neodvisen od
frekvence modulacije vpadne svetlobe. Vendar to še ne pomeni, da lahko moduliramo s poljubno 
veliko frekvenco. Poleg relaksacijskega časa detektorja ima namreč karakteristični čas tudi
elektronsko vezje, ki določa zgornjo mejo za frekvenco. Ta je enak $\tau_e = R_eC_e$, pri čemer
sta $R_e$ upornost sistema in $C_e$ električna kapaciteta detektorja. 
\begin{figure}[h]
\centering
\def\svgwidth{65truemm} 
\input{slike/11_Piro.pdf_tex}
\caption{Spektralni odziv piroelektričnega detektorja na eni strani določajo toplotne izgube 
$\Lambda$ in toplotna kapaciteta detektorja $C$, navzgor pa odziv omejuje odziv elektronskega vezja $\tau_e$.}
\label{fig:Piro}
\end{figure}

\begin{definition}
Piroelektrični detektor naredimo iz kristala LiTaO$_3$ s koeficientom piroelektričnosti
$a = 2,3 \times 10^{-4}~\si{\ampere \second /\metre^2 \kelvin}$ in povprečno 
dielektričnostjo $\varepsilon = 50$. Izračunaj dovoljeno električno upornost sistema, 
če želimo, da detektor deluje za frekvence do 1~MHz. 
Dimenzija detektorja je $S = 1~\si{\centi\metre^2}$ in debelina $d = 1~\si{\milli\metre}$.
\end{definition}

\section{Fotoefekt}
Delovanje kvantnih detektorjev temelji na fotoefektu. To je pojav, pri katerem vpadli
fotoni iz snovi izbijajo elektrone. Izbiti elektroni lahko ubežijo kot prosti elektroni
(t. i. zunanji fotoefekt), ali pa ostanejo ujeti v snovi -- a mobilni -- in tako povečajo 
njeno prevodnost (notranji fotoefekt). V obeh primerih pride do fotoefekta le, 
če je energija vpadlih fotonov večja od neke določene energije.
Pod to vrednostjo fotoefekta ni, ne glede na moč vpadne svetlobe.
Fotoefekt je prvič opazil Hertz\footnote{Nemški fizik Heinrich Hertz, 1857--1894.} 
leta 1887, za njegovo razlago leta
1905 pa je Einstein\footnote{Nemški fizik in nobelovec Albert Einstein, 1879--1955.} 
dobil nobelovo nagrado. 

Poglejmo najprej zunanji fotoefekt, pri katerem elektron postane povsem prost. 
Da se to sploh lahko zgodi, mora biti energija vpadlega fotona dovolj velika, da 
elektron premaga potencialno bariero in izstopi iz prevodnega pasu (slika~\ref{fig:Nivoji}\,a). 
Najmanjšo energijo, ki je za to potrebna, imenujemo v kovinah izstopno delo. 
Če je energija fotona večja, gre preostanek energije v kinetično energijo izbitega
elektrona.

\begin{figure}[h]
\centering
\def\svgwidth{140truemm} 
\input{slike/11_Nivoji.pdf_tex}
\caption{Shema energijskih pasov in zunanjega fotoefekta v kovini (a) in polprevodniku (b) ter
notranjega fotoefekta v polprevodniku (c). $\Phi$ označuje izstopno delo, $E_g$ širino reže 
med valenčnim in prevodnim pasom poprevodnika,
$E_a$ pa elektronsko afiniteto. }
\label{fig:Nivoji}
\end{figure}

Zunanji fotoefekt poteka tudi v polprevodnikih (slika~\ref{fig:Nivoji}\,b). 
V tem primeru foton izbije elektron iz valenčnega pasu, njegova energija pa mora biti 
večja od vsote energije reže in elektronske afinitete, da lahko elektron zapusti snov. 
Z uporabo ustreznih materialov lahko dosežemo, da je elektronska afiniteta
negativna in je zato potrebna energija fotona kar enaka širini energijske reže.

Izstopno delo za kovine $\Phi$ je od okoli 2~eV za cezij pa do okoli 6~eV za platino. 
Ustrezna valovna dolžina svetlobe, ki še povzroči fotoefekt, je tako 
\beq
\lambda \leq \frac{hc}{\Phi},
\eeq
kar je 580~nm za primer cezija in samo okoli 200~nm za platino. Če želimo fotoefekt
izkostisti za detektorje vidne svetlobe, uporabimo druge snovi,
na primer Cs-Te, Cs-Sb, Na-K-Sb-Cs ali GaAs:Cs. 
Tako lahko zaznavamo fotone z valovnimi dolžinami od ultravijolične svetlobe 
pa vse do bližnje infrardeče. 

Pri notranjem fotoefektu (slika~\ref{fig:Nivoji}\,c) elektron snovi ne zapusti, ampak zgolj preide iz enega 
energijskega pasu v drugega. Tipično to poteka v polprevodnikih, kjer absorpcija fotona 
povzroči nastanek para elektron-vrzel, prag za nastanek para pa določa širina reže med 
energijskima nivojema. Primeri detektorjev, ki temeljijo na zunanjem fotoefektu, so 
fotocelice in fotopomnoževalke, na notranjem fotoefektu pa temeljijo na primer
fotoprevodniki, polprevodniške in plazovne fotodiode.

Za zdaj smo napisali, da fotoefekt poteče, ko foton izbije elektron. Vendar pri tem ni 
uspešen prav vsak foton, zato vpeljemo še en parameter, ki ga imenujemo kvantni izkoristek $\eta$.
Ta parameter pove verjetnost, da vpadli foton z valovno dolžino $\lambda$ oziroma frekvenco $\nu$ iz 
snovi izbije elektron. Električni tok, ki steče pri vpadni svetlobni moči $P$, je tako
\boxeq{11:eta}{
I = \eta e_0 n_F = \eta \frac{e_0 P}{h \nu},
}
kje je $n_F$ število vpadnih fotonov na časovno enoto.
Kvantni izkoristek je močno odvisen od valovne dolžine vpadne svetlobe in seveda
od snovi, na katero svetloba vpada. Za fotone z energijo, ki je manjša od izstopnega 
dela oziroma od širine energijske reže, je kvatni izkoristek praktično enak nič, 
nato pa strmo naraste in lahko doseže vrednosti, večje od $90~\%$. Podrobneje ga bomo 
obravnavali pri posameznih primerih detektorjev.

\begin{remark}
V praksi ločimo dve vrsti kvantega izkoristka: zunanji in notranji. Zunanji je vpeljan kot 
razmerje števila izbitih elektronov in fotonov, ki vpadejo na detektor. Ker pa se 
ob vpadu na detektor vedno nekaj fotonov odbije ali siplje, vpeljemo še notranji kvatni 
izkoristek kot razmerje števila elektronov in fotonov, ki se dejansko absorbirajo v detektorju.
Zunanji izkoristek je vedno manjši od notranjega in je neke vrste efektivni izkoristek.
\end{remark}

Iz enačbe(\ref{11:eta}) hitro izračunamo še občutljivost detektorja 
\boxeq{11:R}{
R = \frac{I}{P} = \frac{\eta e_0 }{h \nu}.
}

\section{Vakuumska fotodioda (fotocelica) in fotopomnoževalka}
\subsection*{Fotocelica}
Najpreprostejši kvantni detektor na zunanji fotoefekt je fotocelica ali vakuumska fotodioda
(slika~\ref{fig:Fotoefekt}). Fotocelica deluje tako, da svetloba vpada na katodo, 
zaprto v vakuumirani stekleni bučki, in tam povzroči fotoefekt. Izbite elektrone 
z zunanjo napetostjo pospešimo do anode in merimo električni tok, ki steče med 
katodo in anodo. Ker je tok sorazmeren s številom vpadlih fotonov, lahko na ta 
način izmerimo moč vpadne svetlobe.

\begin{figure}[h]
\centering
\def\svgwidth{60truemm} 
\input{slike/11_Fotoefekt.pdf_tex}
\caption{Shema fotocelice, v kateri poteka fotoefekt. 
Vpadna svetloba iz kovinske katode izbije elektrone, zaradi česar med katodo 
in anodo steče tok.}
\label{fig:Fotoefekt}
\end{figure}

Območje detekcije fotocelice je določeno z izstopnim delom kovine, iz katere fotoni izbijajo elektrone. 
Potrebno energijo fotona lahko precej zmanjšamo, če namesto čistih kovin uporabimo bi- ali 
večalkalne katode (npr. Na$_2$KSbCs), ali pa polprevodnike, na katere nanesemo tanko plast 
Cs ali Cs$_2$O. Tako ustvarimo negativno elektronsko afiniteto in izstopno delo je enako širini energijske
reže. Na ta način lahko zaznavamo svetlobo do valovnih dolžin okoli $1600~\si{\nano\metre}$. 
Na ultravijoličnem območju je delovanje
omejeno na okoli $160~\si{\nano\metre}$ zaradi neprepustnosti stekla, iz katerega je narejena bučka.
\begin{figure}[h]
\centering
\def\svgwidth{130truemm} 
\input{slike/11_SpekterKatode.pdf_tex}
\caption{Kvantni izkoristek fotocelic za različne snovi. Povzeto po Hamamatsu Photonics.}
\label{fig:Fotodioda}
\end{figure}

Čas odziva vakuumske fotodiode je odvisen od časa preleta elektronov od katode do anode. 
Da je ta čas čim krajši, je napetost na fotocelici pogosto več kV. Tedaj lahko dosežemo 
zelo kratke odzivne čase, tudi do 0,1~ns. Enostavnost in hitrost sta torej prednosti fotocelice, 
njena glavna pomanjkljivost pa je razmeroma nizek kvantni izkoristek. 
Ta je seveda močno odvisen od valovne dolžine vpadlega valovanja in snovi, iz 
katere je narejena katoda. Največje vrednosti, ki jih dosega, so okoli $40~\%$, 
pogosto pa več velikostnih redov manj. 
Vrednosti so razmeroma nizke, saj se izbiti elektroni gibljejo v vse
smeri in se pogosto sipljejo, preden sploh dosežejo površino. 

Dodaten problem fotocelic je, da pri končnih temperaturah prihaja do spontane oddaje elektrona.
Nekaj električnega toka zato teče tudi v popolni temi. To je rako imenovani temni 
tok fotodiode in tipično dosega vrednosti okoli $10^{-15}~\si{\ampere}$, lahko pa tudi do več nA. 
Za občutljive meritve je treba zato vakuumsko fotodiodo hladiti. 

\vskip1cm
\begin{definition}
Izračunaj občutljivost fotocelice na osnovi GaAs za valovanje z valovno dolžino $\lambda=620$~nm.
Pri tem kvantni izkoristek odčitaj s slike~(\ref{fig:Fotodioda}).
\end{definition}

\subsection*{Fotopomnoževalka}
Fotopomnoževalke so fotocelice z vgrajenim ojačenjem. Ojačenje dosežemo tako, da 
izbit fotoelektron najprej pospešimo z napetostjo $100$--$150~\si{\volt}$ na vmesno elektrodo, 
tako imenovano dinodo, iz katere izbije več ($\sim 5$--$10$, redkeje tudi do 40) 
sekundarnih elektronov. Ti elektroni
potujejo do naslednje dinode, ki je pod višjo pozitivno napetostjo (tipično okoli $100~\si{\volt}$
višjo), kjer ponovno izbijejo elektrone, ki vpadejo na naslednjo dinodo, ki je pod še višjo napetostjo ... 
To pomnoževanje se večkrat ponovi (navadno okoli desetkrat),
število elektronov eksponentno narašča in na en vpadli foton lahko dobimo $10^9$ elektronov na anodi. 
Občutljivost fotopomnoževalk je tako precej večja od občutljivosti vakuumske fotodiode in
dosega odzivnost na anodi do $R\sim 10^6~\si{\ampere/\watt}$.
Fotopomnoževalka tako omogoča štetje posameznih fotonov, po drugi strani pa moramo pri 
običajnih osvetlitvah paziti, da fotopomnoževalke ne osvetlimo preveč. 
\begin{figure}[h]
\centering
\def\svgwidth{80truemm} 
\input{slike/11_PMT.pdf_tex}
\caption{Shema fotopomnoževalke. Vpadna svetloba iz katode izbije elektrone, ti pa 
iz dinod izbijajo dodatne elektrone in izhodni signal se močno ojača.}
\label{fig:PMT}
\end{figure}

Fotopomnoževalke imajo zelo kratek odzivni čas, ki je odvisen od postavitve dinod. Posamezni 
elektroni do anode potujejo različno dolgo, zato je sunek na izhodu 
razširjen, tipično okoli $\sim 0,1$--$20~\si{\nano\second}$.  
Za manj zahtevne aplikacije pogosto merimo kar povprečni tok z anode. Kadar pa opazujemo
posamezne fotone, zaznamo na izhodu zaporedje sunkov. Takrat lahko 
amplituda izhodnega signala močno niha, saj je koeficient ojačenja 
odvisen od števila izbitih elektronov, kar pa je statistični proces. 

\section{Fotoprevodni detektorji}
Fotoprevodni detektorji\footnote{Fotoprevodne detektorje včasih imenujejo tudi fotouporniki.} 
so detektorji, ki temeljijo na notranjem fotoefektu.
Vpadli foton z dovolj veliko energijo se absorbira, vendar ne izbije elektrona v prostor, 
ampak ga iz valenčnega pasu dvigne v prevodnega. Pri tem nastane par elektron-vrzel. 
Ob priključeni napetosti se ti nosilci naboja začnejo premikati in steče tok, 
ki ga merimo. Z naraščajočim številom fotonov se prevodnost fotoprevodnika veča, 
zato lahko z merjenjem upornosti določimo 
intenziteto vpadle svetlobe. Tipično so fotoprevodniki iz polprevodnikov, 
lahko pa so tudi iz izolatorjev. 

Da foton lahko vzbudi elektron iz valenčnega v prevodni pas, mora biti njegova energija dovolj velika. 
V čistih (nedopiranih) polprevodnikih to pomeni, da mora biti energija fotona večja od 
širine reže. Za silicij, na primer, je širina reže 1,1~eV, s čimer lahko zaznavamo svetlobo
z valovno dolžino do okoli 
$1,1~\si{\micro\meter}$, za germanij 0,67~eV ($1,8~\si{\micro\meter}$) in za PbS 0,37~eV
($3,4~\si{\micro\meter}$). Za detekcijo daljših valovnih dolžin ne uporabljamo polprevodnikov
z manjšo energijsko režo, ampak dopirane polprevodnike (slika~\ref{fig:FPrevodnik}). 
Z novim energijskim nivojem med valenčnim in prevodnim pasom občutno zmanjšamo 
potrebno energijo vpadlih fotonov. Vendar je pri teh nizkih energijah prispevek termično 
vzbujenih elektronov že tako velik, da je treba detektorje hladiti, navadno s tekočim
dušikom ali celo tekočim helijem. Tak primer je germanij, dopiran s cinkom, 
s katerim lahko zaznavamo svetlobo do okoli $40~\si{\micro\meter}$. Pri tem ga hladimo
na $4~\si{\kelvin}$, da zmanjšamo pojav termično vzbujenih nosilcev naboja. 
\begin{figure}[h]
\centering
\def\svgwidth{150truemm} 
\input{slike/11_FPrevodnik.pdf_tex}
\caption{Shema prehoda elektrona v fotoprevodniku: prehod v čistem polprevodiku (a), 
$n$-dopiranem polprevodniku (b) in $p$-dopiranem polprevodniku (c). 
Z dopiranjem povečamo območje delovanja detektorja v infrardeče območje. }
\label{fig:FPrevodnik}
\end{figure}

Izračunajmo električni tok, ki steče, ko posvetimo na fotoprevodnik. Spomnimo se, da
je gostota električnega toka $j$ enaka vsoti prispevkov elektronov in vrzeli
\beq
j = e_0 n_v v_v + e_0 n_e v_e,
\eeq
pri čemer $n_v$ in $n_e$ pomenita gostoto vrzeli in elektronov v snovi, $v_v$ in $v_e$ pa 
hitrost vrzeli in elektronov. Ta je sorazmerna z električno poljsko jakostjo $E$, ki je priključena
 na vzorec, sorazmernostni faktor pa je gibljivost $\beta$. Ko posvetimo na vzorec, 
 se $n_v$ in $n_e$ povečata za $\Delta n_v$ in $\Delta n_e$,
gostota električnega toka pa naraste za
\beq
\Delta j = e_0 \Delta n_v v_v + e_0 \Delta n_e v_e.
\label{FP_j}
\eeq
V stacionarnem primeru se število nosilcev naboja ne spreminja in velja
\beq
0 = \frac{dn_v}{dt} = \frac{\eta_v P}{h \nu (Sl)} - \frac{\Delta n_v}{\tau_v}
\eeq
in podobno za elektrone. Pri tem je $\eta$ kvatni izkoristek, $P$ moč vpadne svetlobe,
$Sl$ prostornina detektorja in $\tau$ življenjski čas vrzeli oziroma elektrona. 
Ko stacionarno vrednost $\Delta n_v$ in $\Delta n_e$ vstavimo v enačbo~(\ref{FP_j}), dobimo
\beq
\Delta j = e_0 \frac{\eta_v P \tau_v}{h \nu (Sl)} \beta_v  E + 
e_0 \frac{\eta_e P \tau_e}{h \nu (Sl)} \beta_e  E.
\eeq
Če vpeljemo še napetost $U = E/l$, zapišemo celotni tok skozi fotoprevodnik zaradi vpadle svetlobe kot
\beq
\Delta I = \Delta j S = \frac{e_0 U P }{h \nu l^2} \left(\eta_v \tau_v \beta_v + 
\eta_e \tau_e \beta_e \right).
\eeq
Pogosto je gibljivost elektronov znatno večja od gibljivosti vrzeli (npr.
$0,135~\si{\meter}^2/\si{\volt\second}$ proti $0,048~\si{\meter}^2/\si{\volt\second}$ za silicij), 
zato prvi člen v oklepaju
zanemarimo in zapišemo
\beq
\Delta I = G \left( \frac{e_0 \eta_e}{h\nu}\right) P,
\eeq
pri čemer je koeficient ojačenja 
\beq
G = \frac{\beta_e \tau_e U}{l^2} = \frac{\tau_e}{\tau}.
\eeq
Vpeljali smo še čas preleta $\tau = v_e/l = \beta_e E/l = \beta_e U/l^2$.

Koeficient $G$ opiše ojačenje signala. Njegova vrednost je odvisna od 
vrste snovi in gibljivosti nosilcev naboja v njej, velikosti
detektorja in tudi priključene napetosti, zato lahko $G$ zavzane vrednosti od manj kot ena pa
vse do $10^6$. 

\begin{definition}
Izračunali smo spremembo toka, če fotoprevodnik osvetlimo s konstantno vpadno močjo. Pokaži, da
 je v primeru periodično spremenljive moči odziv enak
 \beq
\Delta I_\omega = G \left( \frac{e_0 \eta_e}{h\nu}\right) \frac{P_\omega}{1+i \omega \tau_e}.
 \eeq
 
\end{definition}

Fotoprevodniki so uporabni na širokem spektralnem območju, od ultra\-vijolične 
do daljne infra\-rdeče svetlobe. V vidnem in bližnjem infrardečem omočju se 
uporablja pretežno silicijeve fotoprevodnike, germanijeve
pa za valovne dolžine do $1,8~\si{\micro\meter}$. Za zaznavanje valovnih dolžin med okoli 
$2~\si{\micro\meter}$ in $7~\si{\micro\meter}$ so najprimernejši InAs, InSb in PbS detektorji, 
pri še daljših valovnih dolžinah pa se uporablja germanij, dopiran z zlatom, bakrom, cinkom, borom ...
Kvatni izkoristek takih detektorjev je razmeroma velik ($\eta = 0,5$ za Ge:Cu), vendar
je lahko faktor ojačenja $G \ll 1$ (npr. $G = 0,03$ za Ge:Hg). 

Hitrost odziva fotoprevodnika je odvisna od časa preleta nosilcev naboja,
ki je določen z geometrijo detektorja, in od karakterističnega časa elektronskega vezja. 
Tipični odzivni časi so okoli mikrosekunde, vendar lahko sežejo
tudi do desetin milisekund, ali pa v izjemnih primerih do nanosekund za zelo majhne detektorje.
S skrajšanjem rekombinacijskega časa lahko sicer skrajšamo odzivni čas detektorja, 
vendar hkrati zmanjšamo tudi njegovo občutljivost.

\begin{remark}
Fotoprevodni detektorji so narejeni iz zelo tankih plasti fotoprevodnika, saj močno absorbira
svetlobo. Tako za absorpcijo $70-90\%$ svetlobe zadošča le $1$--$2~\si{\micro\meter}$ debela plast.
Elektrode se pogosto prepletajo, da se zmanjša dolžina preleta $l$ in poveča ojačenje $G$. 
\end{remark}

\section{Polprevodniške fotodiode}
Drugi primer detektorjev, ki temeljijo na notranjem fotoefektu, so polprevodniške fotodiode.
Te so danes najpogostejša in najbolj razširjena vrsta detektorjev svetlobe, uporabljamo jih med
drugim tudi v fotoaparatih in sončnih celicah. Fotodiode so sestavljene iz $p$- in $n$-dopiranega 
polprevodnika ($p$-$n$ fotodiode) ali pa je med njima še plast nedopiranega (intrinzičnega) 
polprevodnika ($p$-$i$-$n$ fotodioda). Ko svetloba vpade na $p$-$n$ (ali $p$-$i$-$n$) 
stik, se fotoni absorbirajo in nastajajo 
pari elektron-vrzel. Nosilci naboji potujejo v različnih smereh, elektroni stečejo v eno smer,
vrzeli pa v nasprotno. Odvisno od načina delovanja lahko izmerimo tok, ki steče skozi 
stik, ali pa napetost, ki se pojavi na stiku. 

Spektralni odziv fotodiod je seveda odvisen od energijske reže polprevodnika, 
iz katerega je fotodioda narejena.
Silicijeve fotodiode so tako uporabne za zaznavanje valovnih dolžin do okoli
$1,1~\si{\micro\meter}$, za
večje valovne dolžine (do $1,6~\si{\micro\meter}$) uporabljamo InGaAs. Izkoristek fotodiod
je navadno zelo velik in presega $50~\%$, pri energiji fotonov blizu energijske reže 
je vrednost izkoristka kar blizu 1.
Za razliko od fotoprevodnikov fotodiode signala
ne ojačujejo, imajo pa praviloma hitrejši odziv, tipično okoli nanosekunde.

Dioda lahko deluje v različnih načinih (slika~\ref{11_PD}). 
Lahko jo priključimo v prevodni smeri, najpogosteje jo priključimo v zaporni smeri, saj je v
tem primeru tok skozi diodo linearno sorazmeren z intenziteto vpadne svetlobe, lahko 
je dioda kratko sklenjena, lahko pa je dioda v odprtem električnem krogu, v t.i. fotovoltaičnem 
načinu. V nadaljevanju bomo vse primere podrobneje spoznali.
\begin{figure}[h]
\centering
\def\svgwidth{140truemm} 
\input{slike/11_diode.pdf_tex}
\caption{Različne vezave fotodiode: v prevodni smeri (a), v zaporni smeri (b), kratko sklenjena (c) in 
v fotovoltaičnem načinu (d)}
\label{11_PD}
\end{figure}

\subsection*{Stik $pn$}
Ponovimo najprej, kaj se zgodi ob stiku $p$- in $n$- tipa polprevodnika. Pri tem tip $p$ označuje
polprevodnik, dopiran s trivalentnimi akceptorskimi primesmi, ki v snovi ustvarijo vrzeli.
Energijski nivo primesi je malo nad vrhom valenčnega pasu, zato je Fermijeva energija
polprevodnika premaknjena navzdol proti valenčnem pasu (slika~\ref{11_PN1}\,a). 
Po drugi strani $n$ tip označuje polprevodnike s petvalentnimi 
donorskimi primesmi, ki v snov prinesejo dodatne elektrone. Njihov energijski nivo je malo 
pod prevodnim pasom, zaradi česar je Fermijeva energija pomaknjena navzgor proti prevodnemu pasu
(slika~\ref{11_PN1}\,b).

Ko staknemo polprevodnik tipa $p$ s polprevodnikom tipa $n$, elektroni 
z območja z višjo koncentracijo (tip $n$) difundirajo v območje z nižjo koncentracijo
(tip $p$), kjer se rekombinirajo z vrzelmi. 
Ob stiku tako nastane ozek pas,  imenujemo ga izpraznjeni sloj, kjer ni več 
prostih nosilcev naboja. Ostanejo pa pozitivno nabiti donorski atomi na strani $n$
in negativno nabiti akceptorski atomi na strani $p$. Ti naboji povzročijo nastanek  
električnega polja, ki kaže od $n$ proti $p$. Nastalo polje zaustavi rekombinacijo, saj odbija
elektrone in vrzeli od stika. V ravnovesju se Fermijeva energija izenači, potencialni
skok pa je približno enak $\Delta E \approx E_d-E_a$, kar je le malo manj od 
širine reže $E_g$ (slika~\ref{11_PN1}\,c).

\begin{figure}[h]
\centering
\def\svgwidth{140truemm} 
\input{slike/11_PN1.pdf_tex}
\caption{Shema energijskih nivojev v $p$- (a) in $n$-tipu (b) polprevodnika ter na $p$-$n$ stiku (c), 
v katerem se Fermijevi energiji izenačita. Med obema polprevodnikoma nastane izpraznjeni sloj, kar 
povzroči nastanek električnega polja.}
\label{11_PN1}
\end{figure}

\begin{figure}[h]
\centering
\def\svgwidth{140truemm} 
\input{slike/11_PNU.pdf_tex}
\caption{Shema energijskih nivojev v $p$-$n$ stiku, ko na stik priključimo napetost
v prevodni smeri (a) in v zaporni smeri (b). Če v izpraznjenem sloju pride do absorpcije
fotona in nastanka para elektron-vrzel, elektron ``zdrsi'' proti strani $n$, vrzel pa proti
strani $p$.}
\label{11_PNU}
\end{figure}

Priključimo zdaj na diodo napetost, tako da je pozitivna na $p$ strani diode. Takrat 
pravimo, da smo na diodo priključili napetost v prevodni smeri. Ker lahko energijske
pasove razumemo kot potencialno energijo elektronov, s priključeno napetostjo
zmanjšamo razliko potencialnih energij in elektroni lažje prehajajo iz $n$ v $p$ del. 
Zaradi zmanjšanja potencialne razlike med $p$ in $n$ stranjo za $e_0U$ pride do povečanja toka 
večinskih elektronov iz $n$ v $p$ za faktor $\exp(e_0 U/kT)$, tok manjšinskih elektronov
iz $p$ v $n$ pa ostaja enak, saj ni ovisen od globine potencialnega skoka 
(slika~\ref{11_PNU}\,a). 

Povsem enak razmislek lahko naredimo, če priključimo 
na $n$ stran pozitivni pol, na $p$ stran pa negativnega, če torej priključimo
napetost v zaporni smeri. V tem primeru potencialna razlika naraste in tok 
večinskih elektonov se zmanjša za faktor $\exp(-e_0 |U|/kT)$, tok 
manjšinjskih elektronov pa ostane nespremenjen (slika~\ref{11_PNU}\,b).

Celotni tok skozi $p$-$n$ stik je sestavljen iz prispevkov elektronov in vrzeli, 
opiše pa ga tako imenovana karakteristična enačba diode (slika~\ref{11_IU})
\boxeq{11:dioda}{
I = I_0 (e^{e_0 U/kT}-1).
}
Pri tem $I_0$ označuje tok manjšinskih nosilcev naboja\footnote{Pravimo
mu tudi zaporni tok, tok nasičenja ali temni tok. Slednje ime sledi iz tega, da
ta tok teče skozi fotodiodo tudi v odsotnosti svetlobe.}
in je navadno zelo majhen. Njegova vrednost je odvisna od snovi, površine
detektorja, poleg tega pa je eksponentno odvisna od temperature. Znaša 
tipično okoli $10^{-5}$-$10^{-15}~\si{\ampere}$, pri čemer najmanjše
vrednosti dosegamo le ob močnem hlajenju. 

\begin{figure}[h]
\centering
\def\svgwidth{100truemm} 
\input{slike/11_IU.pdf_tex}
\caption{$I(U)$ karakteristika neosvetljene fotodiode (modra črta)
in osvetljene fotodiode (rdeče črte). Naraščajoča intenziteta vpadne svetlobe
krivuljo premakne navzdol. S simboli so označene točke delovanja za različne načine.
}
\label{11_IU}
\end{figure}
 
\subsection*{Delovanje fotodiode}
Ko na polprevodnik vpade foton, ki ima energijo večjo od širine reže, 
lahko vzbudi elektron iz valenčnega v prevodni pas in nastane par elektron-vrzel. 
Če se to zgodi v izpraznjenem sloju $p$-$n$ stika, steče elektron pod vplivom 
električnega polja na stran $n$, vrzel pa na stran $p$ (slika~\ref{11_PNU}\,c). 
Premik nosilcev naboja, do katerega je prišlo zaradi absorpcije fotona, 
torej vedno steče v zaporni smeri. 
Njegova velikost je odvisna od moči vpadne svetlobe in jo lahko zapišemo kot  
\boxeq{11:if}{
I_f = e_0 \eta n_F = e_0 \frac{\eta P}{h \nu},
}
pri čemer smo z $n_F$ onačili število vpadnih fotonov na časovno enoto, 
$\eta$ je kvantni izkoristek,
$P$ pa označuje moč vpadne svetlobe. Celoten tok skozi fotodiodo je kombinacija diodnega 
in svetlobnega toka, zato karakteristiko fotodiode zapišemo kot 
\boxeq{11:fotodioda}{
I = I_0 (e^{e_0 U/kT}-1) - I_f.
}
Vpadna svetloba torej povzroči zmanjšanje električnega toka skozi diodo, 
kar na sliki~(\ref{11_IU}) predstavlja premik karakteristike diode v vertikalni 
smeri (rdeče črte). Naraščajoča intenziteta svetlobe premika krivuljo proti 
bolj negativnim vrednostim tokov. 

Prvi način delovanja fotodiode, ki ga bomo obravnavali, je {\bf fotovoltaični način}.
To je način, pri katerem električni tokokrog ni sklenjen (slika~\ref{11_PD}\,d), 
zato ob absorpciji fotona in nastanku
para elektron-vrzel tok ne more steči. Še vedno pa se izbiti elektron pod vplivom električnega polja
na stiku premakne proti $n$ delu, vrzel pa proti $p$. Na diodi se tako pojavi napetost, 
katere vrednost lahko izračunamo iz karakteristične enačbe diode, če upoštevamo, da je $I=0$. Sledi
\beq
U_p = \frac{kT}{e_0}\ln \left(1+ \frac{I_f}{I_0}\right).
\eeq
Pri večji intenziteti vpadle svetlobe, ko se krivulja na grafu (\ref{11_IU}) pomika navzdol, se rešitev
gornje enačbe po abscisi premika proti desni. Večja intenziteta vpadne svetlobe torej pomeni večjo 
pozitivno napetost na diodi, zato tudi odzivnost v tem primeru merimo v $\si{\volt}/\si{\watt}$.
Pri dovolj velikih vpadnih močeh je zveza med vpadno močno in fotonapetostjo
logaritemska. Taka vezava fotodiode tako omogoča zaznavanje vpadne moči v zelo širokem intervalu,
najpogosteje pa se ta način delovanja uporablja v sončnih celicah. 

Drugi način delovanja je {\bf kratko sklenjena} fotodioda (slika~\ref{11_PD}\,c). V tem primeru je 
napetost na diodi enaka nič in je tok skozi tokokrog kar enak toku zaradi vpadle svetlobe $I_f$
(slika~\ref{11_IU}).

Najbolj splošno uporabljen način za detekcijo svetlobe je način, v katerem napetost na diodo 
priključimo {\bf v zaporni smeri} (slika~\ref{11_PD}\,b).
Takrat se tok skozi diodo spreminja linearno z močjo vpadne svetlobe, odziv pa je hitrejši
kot pri kratko sklenjeni diodi. Če dodamo v tokokrog zaporedno vezan še nek upornik, se odziv
spremeni. Po grafu (slika~\ref{11_IU}) se ne premikamo več navzdol, ampak pod kotom proti desni.
Enačbo, ki opisuje premico, ki seka karakteristične krivulje, lahko preprosto zapišemo
z Ohmovim zakonom $U = -|U_0|-RI$. 

Prednosti tega načina merjenja je več. Zaradi priključene napetosti se zmanjša čas preleta
nosilcev naboja in posledično se zmanjša odzivni čas detektorja. Dodatno se poveča
širina izpraznjenega pasu, kar zmanjša kapaciteto stika (stik $p$-$n$ namreč deluje kot neke vrste 
kondenzator in časovni odziv je odvisen od njegove kapacitete) in s tem odzivni čas. Povečana
izpraznjena plast pa vodi do večjega območja, v katerem lahko pride do absoprcije fotonov. 

Povejmo še nekaj o zgradbi fotodiode. Shema preproste fotodiode je prikazana na sliki~(\ref{11_shema}\,a).
Na dnu je elektroda, sledi plast $n$, nad njo je tanka plast $p$, na katero vpada svetloba.
Bistveno je, da je osvetljena plast tanka, da svetloba lahko prodre v bližino stika. Zato so 
debeline zgornje plasti tipično submikronske. Dodatno na fotodiode pogosto nanesemo
še dodatno antirefleksijsko plast (SiO$_2$). Fotoobčutljiv del komercialnih fotodiod
je tipično od nekaj $100~\si{\micro\meter}^2$ pa do več $100~\si{\milli\metre}^2$. Pri 
tem imajo večje diode seveda počasnejši odziv. 

\begin{remark}
Poleg do zdaj obravnavanih fotodiod, poznamo tudi heterostrukturne fotodiode, kjer sta $p$ in $n$ del
narejena iz druge snovi. Poseben primer so Schottkyjeve fotodiode, kjer eno plast polprevodnika
nadomestimo z zelo tanko plastjo kovine (slika~\ref{11_shema}\,c). Te so uporabne predvsem pri 
visokih energijah (v UV območju), 
saj je v navadnih fotodiodah absorpcija za te valovne dolžine prevelika, na površini pride do 
rekombinacije in zmanjšanja kvantnega izkoristka. Poleg tega je odziv Schottkyjevih fotodiod zelo hiter, 
ker nizka upornost kovine občutno zmanjša $RC$ konstanto stika. Odzivni časi dosegajo pikosekundne vrednosti. 
\end{remark}

\begin{figure}[h]
\centering
\def\svgwidth{140truemm} 
\input{slike/11_shema.pdf_tex}
\caption{Shema $p$-$n$ fotodiode (a), $p$-$i$-$n$ fotodiode, ki se od navadne 
$p$-$n$ razlikuje po vmesni plasti intrinzičnega
polprevodnika (b) in shema Schottkyjeve fotodiode (c)}
\label{11_shema}
\end{figure}

\subsection*{Fotodioda $pin$}
Fotodiode $p$-$i$-$n$ se od navadnih $p$-$n$ razlikujejo po tem, da med $p$- in $n$-plast 
vključimo še plast nedopiranega polprevodnika (slika~\ref{11_shema}\,b). S tem se efektivno bistveno poveča območje
izpraznjene plasti in njegova širina postane praktično neodvisna od priključene napetosti.
Povečanje izpraznjene plasti omogoča zaznavanje bistveno večjega deleža vpadne svetlobe, 
poleg tega pa zmanjša kapaciteto stika in s tem njegovo $RC$ konstanto. Slabost dodatnega
sloja je povečanje časa preleta čez izpraznjeno plast, vendar lahko
z ustrezno optimizacijo konstrukcije dosežemo odzivne čase nekaj deset ps.

\begin{figure}[h]
\centering
\def\svgwidth{100truemm} 
\input{slike/11_SpekterFD.pdf_tex}
\caption{Kvantni izkoristek nekaterih $p$-$i$-$n$ in Schottkyjevih fotodiod.}
\label{11_odziv}
\end{figure}

\section{Plazovne fotodiode}
Ko smo risali karakteristiko fotodiode, nismo narisali popolne slike. 
Pri velikih negativnih napetostih se namreč karakteristika znatno spremeni (slika~\ref{11_plaz}), 
česar ne moremo popisati s preprosto enačbo. Pri zapornih napetostih, ki za nekajkrat presegajo 
širino energijske reže (tipično okoli $10^7~\si{\volt}/\si{\meter}$), 
pride do naglega povečanja električnega toka. Ob absopciji fotona nastali
mobilni nosilci naboja se namreč v električnem polju tako pospešijo, da s trki ustvarjajo nove pare 
elektron-vrzel. Ti novonastali pari ponovno ustvarjajo pare in pride do ``plazu'', podobno kot v 
fotopomnoževalki. En foton torej sproži cel plaz elektronov, zato pravimo, da je plazovna dioda
fotodioda z notranjim ojačenjem. Pri tem je faktor ojačenja tipično $30$-$300$ in jih lahko 
uporabimo za detekcijo posameznih fotonov. Slabost je, da je faktor ojačenja odvisen od
temperature in je zato za natančne meritve potrebna temperaturna stabilizacija.

Napetost, pri kateri deluje plazovna fotodioda, je priključena v zaporni smeri jih jo 
držimo tik pod prebojno napetostjo. Ker že  majhna odstopanja v napetosti povzročijo veliko
spremembo v toku, moramo tudi napetost držati kar se da stabilno. Le to omogoča
linearen odziv fotodiode od moči vpadne svetlobe. Take fotodiode so praviloma zelo hitre 
($50~\si{\pico\second})$ in zelo občutljive. Z ojačenjem signala se ojača tudi šum, a je ta 
porast pogosto manjša kot bi bil prispevek k šumu na zunanjih elektonskih ojačevalcih. 

\begin{figure}[h]
\centering
\def\svgwidth{60truemm} 
\input{slike/11_plaz.pdf_tex}
\caption{Karakteristika plazovne fotodiode}
\label{11_plaz}
\end{figure}

\section{CCD in CMOS detektorji}
Do zdaj smo obravnavali detektorje, ki zaznavajo pretok vpadnih fotonov in spreminjanje
tega pretoka s časom. Dodatno informacijo dobimo, če več fotodetektorjev sestavimo v 
dvodimenzionalno matriko, saj lahko ti detektorji hkrati zaznavajo količino vpadne svetlobe 
iz različnih delov prostora in dobljene podatke sestavimo v sliko. Pri tem en detektor
podaja informacijo o številu vpadnih fotonov v dani časovni enoti (integracijski čas
oziroma čas osvetlitve) en slikovni 
element -- piksel. Slikovni detektorji z veliko ločljivostjo so tako sestavljeni iz 
več milijonov ali celo milijarde posameznih polprevodniških detektorjev in so 
nepogrešljivi v fotoaparatih, video-kamerah, mikroskopiji in astronomiji.

Podrobneje bomo obravnavali dva primera matričnih detektorjev, to so CCD 
({\it charge-coupled-device})\footnote{Za izum CCD detektorjev sta Willard 
S. Boyle in George E. Smith  leta 2009 prejela Nobelovo nagrado.} 
in CMOS ({\it complementary metal-oxide-semicoductor})
\footnote{Teh dveh oznak za detektorje praviloma ne prevajamo. Opisujeta 
strukturo in delovanje naprave in nista vezani zgolj na detekcijo svetlobe.}. Obe vrsti
detektorjev sta si glede zaznavanja svetlobe zelo pododbni, razlika
je predvsem v načinu, kako iz posameznega detektorja dobimo podatek o številu 
vpadlih fotonov oziroma številu vzbujenih elektronov.

\begin{remark}
Slikovni detektorji so seveda lahko sestavljeni tudi iz drugih svetlobnih detektorjev, 
ki smo jih obravnavali v prejšnjih razdelkih. Lahko so iz mikrobolometrov
ali fotoprevodnikov (za IR svetlobo), Schottkyjevih fotodiod (npr. PtSi, ki seže od UV do
okoli $6~\si{\micro\meter}$) ali plazovnih fotodiod.
\end{remark}

\subsection*{CCD}
Detektorjev CCD so sestavljeni iz posameznih tako imenovanih MOS ({\it metal-oxide-semiconductor}
-- kovina-oksid-polprevodnik) kodenzatorjev. Njihova osnova je dopiran silicij, vmesna 
plast med polprevodnikom in prevodno elektrodo pa je navadno zelo tanka plast (pod 100 nm)
SiO$_2$ (slika~\ref{11_MOS}).
Prevodna elektroda je bila prvotno iz kovine (npr. aluminija) in je elementu detektorja dala tudi ime.
Danes je kovino večinoma nadomestil polikristalni silicij (polisilicij), ime pa je ostalo.
Tipična dolžina stranice posameznega elementa znaša okoli $5$-$40~\si{\micro\meter}$. 
\begin{figure}[h]
\centering
\def\svgwidth{140truemm} 
\input{slike/11_MOS.pdf_tex}
\caption{Shema MOS strukture (a), ki je osnova za vse CCD in CMOS slikovne detektorje. Osnova je 
polprevodnik ($p$), na katerem je plast dielektrika (SiO$_2$), na njej pa elektroda (siva). 
Ob absorpciji svetlobe se pojavijo fotoelektroni, katere pozitivna napetost
na elektrodi drži ujete v potencialno jamo (vijolična). Prenos elektronov v CCD detektorju (b).}
\label{11_MOS}
\end{figure}

Ko foton vpade na element MOS skozi tako prozorno elektrodo, v polprevodniku ustvari
par elektron-vrzel. Pozitivna napetost na elektrodi elektrone privlači, vendar jih 
vmesna plast izolatorja tik pod površino ustavi in elektroni tako ostanejo ujeti v potencialni jami. 
Pri tem je število ujetih elektronov sorazmerno številu vpadlih fotonov v času zajemanja slike, 
pomnoženih s kvantnim izkoristkom pri dani valovni dolžini. 

S spreminjanjem napetosti na posameznih elektrodah lahko naboj, ki se lokalno nabere 
v plasti pod izolatorjem v danem času, postopoma prenesemo od posameznega 
piksla do izhodne stopnje. Najprej poteka prenos iz enega elementa na drugega v eni vrstici, 
nato pa še po celotnem stolpcu (slika~\ref{11_CCD}\,a). Na koncu signal sproti ojačujemo, 
pretvorimo v napetost, to pa v digitalni zapis. Pri tem številu elektronov iz posameznega
slikovnega elementa določimo digitalno vrednost glede na barvno globino. 8-bitni zapis slike
tako vsakemu elementu priredi vrednost od 0 do 255, 16-bitni pa od 0 do 65535.

Delovanje detektorjev CCD torej temelji na zaporednem odčitavanju števila fotoelektronov v posameznem 
slikovnem elementu. Ta način je razmeroma počasen in omejuje hitost zajemanja slike. Med 
prenašanjem nabojev do izhoda namreč slike ne moremo zajemati, saj bi prišlo do popačenja signala. 
Ta problem se večinoma rešuje tako, da le del celotnega zaslona zajema svetlobo, drug del
pa je namenjen pretakanju elektronov in omogoča nemoteno praktično neprestano zajemanje slike.
Ker se s tem količina zajete svetlobe zmanjša, se na vsak element doda lečo, ki svetlobo zbere
na detektor. S tem postanejo slikovni detektorji CCD hitrejši in bolj občutljivi. Poleg
tega jih odlikuje tudi razmeroma nizek šum, ki se ga da s hlajenjem še dodatno 
zmanjšati. 

\begin{remark}
Pogosto pri eksperimentu ne potrebujemo največje možne ločljivosti, ki jo nudi detektor. 
Takrat se pogosto poslužujemo združevanja sosednjih elementov, t. i. bininga, na primer $2\times2$
ali $4\times4$, in s tem sicer zmanjšamo ločljivost, ampak hkrati skrajšamo čas
zajemanja slike in zmanjšamo razmerje signal proti šumu. 
\end{remark}

\begin{figure}[h]
\centering
\def\svgwidth{80truemm} 
\input{slike/11_CCD.pdf_tex}
\caption{Shema CCD (a) in shema CMOS (b) slikovnega detektorja. Puščice označujejo premikanje
fotoelektronov.}
\label{11_CCD}
\end{figure}

\subsection*{CMOS}
Osnovni element detektorjev CMOS je enak kot za detektorje CCD (slika~\ref{11_MOS}\,a). 
Bistvena razlika je v načinu zajemanja fotoelektronov. Pri detektorjih CCD je bilo branje 
fotoelektronov zaporedno, pri detektorjih CMOS pa poteka branje vseh slikovnih elementov 
hkrati, pri čemer ima vsak piskel tudi svoj ojačevalnik (slika~\ref{11_CCD}\,b).
Zaradi sprotnega odčitavanja vseh pikslov naenkrat so detektorji CMOS bistveno hitrejši 
od CCD. Odlikuje jih tudi nizka poraba energije in nizka cena. Njihova poglavitna slabost
je večji šum in manjša občutljivost, saj del zaslona, kjer so ojačevalniki, slike ne more
zajemati. 

\subsection*{Barvno zajemanje slik}
Detektorji zaznavajo samo število vpadlih fotonov oziroma bolj natančno število fotoelektronov.
Za nastanek barvne slike moramo vpadle fotone ločiti še po valovni dolžini, kar naredimo
z barvnimi filtri. Namesto enega elementa, ki bi podal informacijo o intenziteti vpadne 
svetlobe, imamo štiri senzorje v kvadratni mreži, rdečega, modrega in dva zelena. 
Večji delež zelenih elementov je zaradi večje občutljivosti človeškega očesa na zeleno barvo. 
Intenziteto svetlobe na posameznem slikovnem elementu dane barve nato odčitamo kot je
opisano zgoraj.
 
\section{Šum pri optični detekciji}
Pri vsakršni detekciji svetlobe je vedno prisoten tudi šum. Beseda šum označuje naključne 
fluktuacije na izhodu iz detektorja, ki jih ne moremo ločiti od signala. Z različnimi 
pristopi lahko šum zmanjšamo, povsem ga pa ne moremo nikoli odpraviti. Obravnava 
šuma je zato najbolj pomembna pri zaznavanju šibkih signalov svetlobe. Pri tem je ključen
parameter najmanjša moč vpadne svetlobe, ki jo še lahko ločimo od šuma, pod to vrednostjo 
pa se signal v šumu izgubi (slika~\ref{11_sum}).
\begin{figure}[h]
\centering
\def\svgwidth{140truemm} 
\input{slike/11_sum.pdf_tex}
\caption{Če je signal velik v primerjavi s šumom, ga na detektorju lahko zaznamo (zgoraj). 
Pod določeno vrednostjo postane velikost signala primerljiva s šumom in signala ne zaznamo več
(spodaj).}
\label{11_sum}
\end{figure}

Na podlagi fizikalnega izvora ločimo več vrst šuma:
\begin{enumerate}
\item šum štetja, do katerega pride zaradi diskretne (kvantne) narave fotonov,
\item termični šum, do katererega pride zaradi termičnih fluktuacij,
\item šum temnega toka, ki predstavlja spontani nastanek para elektron-vrzel oziroma spontano
emisijo elektronov in
\item šum sevanja ozadja.
\end{enumerate}

\subsection*{Šum štetja} 
Ko govorimo o svetlobi, ne smemo pozabiti, da je svetloba sestavljena iz diskretnih fotonov. 
Fotoni vpadajo na detektor posamezno in enkrat jih vpade malo več, 
drugič malo manj. Vpadna moč je zato dejansko povprečna moč $\overline{P}$ in število 
vpadlih fotonov na časovno enoto je povprečna vrednost števila vpadlih fotonov na časovno enoto
\beq
\overline{n} = \frac{\overline{P}}{h\nu}.
\eeq
Pri vpadu fotonov gre za diskretne in neodvisne procese, zato za njihov vpad velja
Poissonova porazdelitev (slika~\ref{11_Poiss}). Verjetnost, da v času $\tau$, ki prestavlja 
čas merjenja, na detektor vpade $N$ fotonov, je tako 
\beq
p(N) = \frac{\overline{N}^N e^{-\overline{N}}}{N!},
\label{Poisson}
\eeq
pri čemer je povprečno število vpadlih fotonov v tem časovnem intervalu 
enako $\overline{N} = \overline{n}\tau$.
\begin{figure}[h]
\centering
\def\svgwidth{90truemm} 
\input{slike/11_poisson.pdf_tex}
\caption{Poissonova porazdelitev verjetnosti za $\overline{N}=2$ (modra), 
$\overline{N}=5$ (rdeča) in $\overline{N}=10$ (zelena). Porazdelitev je 
diskretna, črta je zgolj vodilo.}
\label{11_Poiss}
\end{figure}

Fluktuacije števila fotonov, ki vpadejo na detektor v danem časovnem 
intervalu, označimo z $\Delta N = N-\overline{N}$. V povprečju je ta vrednost seveda enaka nič, 
zato sta bolj merodajni količini varianca, ki je enaka (glej nalogo~\ref{nal:Poiss})
\beq
\sigma^2 = \overline{\Delta N^2}= \overline{(N-\overline{N})^2} = \overline{N},
\label{varianca}
\eeq
in standardni odklon
\beq
\sigma = \sqrt{\overline{\Delta N^2}} = \sqrt{\overline{N}}.
\label{sigma}
\eeq
\begin{definition}
Pokaži, da je povprečje Poissonove porazdelitve (enačba~\ref{Poisson}) vedno pri $N = \overline{N}$
in standardni odklon $\sigma = \sqrt{\overline{N}}$.
\label{nal:Poiss}
\end{definition}

Standardni odklon, ki je merilo za velikost šuma, torej narašča z naraščajočim številom 
vpadnih fotonov $\overline{N}$. Vendar nas absolutni šum večinoma ne zanima, 
saj je pri detekciji ključno razmerje 
signala proti šumu. To količino označimo s $SNR$ ({\it Signal to Noise Ratio}-- razmerje med 
signalom in šumom)\footnote{Pogosto se uporablja tudi oznako $S/N$. Tukaj smo jo 
zaradi jasnosti zamenjali, saj $N$ označuje število fotonov oziroma elektronov.}. 
V primeru Poissonove porazdelitve in šuma štetja velja
\boxeq{11:SNR}{
SNR = \frac{\overline{N}}{\sigma} = \sqrt{\overline{N}}.
}
Vidimo, da razmerje signala proti šumu narašča z naraščajočim številom vpadlih fotonov, 
relativni šum pa ob večji vpadni moči svetlobe pojema. Za primer poglejmo dva primera vpadne svetlobe. 
V prvem je povprečno število vpadlih fotonov v danem časovnem 
intervalu $10^6$, v drugem pa le $100$. Pri vpadu močnejšega signala 
na detektorju zaznavamo $10^6 \pm 1000$ fotonov, pri vpadu šibkejšega
pa $100 \pm 10$. Čeprav je absolutni šum v prvem primeru znatno večji, 
je relativni šum stokrat manjši. Za zmanjšanje šuma štetja mora biti 
torej signal kar se da velik. 

Pri šumu štetja gre za osnovno značilnost svetlobe, zato je ta vrsta šuma
prisotna pri prav vseh načinih detekcije. Mi si podrobeneje oglejmo, 
kako se ta šum izraža pri detekciji s fotodiodami. 

Naj svetloba s povprečno močjo $\overline{P}$ vpada na fotodiodo.
Povprečno število fotoelektronov, ki se pojavijo v časovnem intervalu 
$\tau$, je kar enako številu vpadlih fotonov, pomnoženim 
s kvantnim izkoristkom. 
\beq
\overline{N}_e = \frac{\overline{P}\tau}{h \nu}\eta.
\eeq
Povprečni tok, ki steče skozi detektor, je  
\beq
\overline{I} = \frac{\overline{N}_e e_0}{\tau},
\eeq
fluktuacije izhodnega električnega toka pa so
\beq
\overline{\Delta I^2}=\overline{(I-\overline{I})^2} = \overline{(N_e-\overline{N}_e)^2}\,
\frac{e_0^2}{\tau^2} = \overline{N}_e\,\frac{e_0^2}{\tau^2}= \overline{I}\,\frac{e_0}{\tau},
\eeq
pri čemer smo upoštevali enačbo~(\ref{varianca}). Vpeljemo še pasovno širino 
detekcije $\Delta\nu_B = 1/(2\tau)$ in dobimo
\boxeq{11:sum}{
\sqrt{\overline{\Delta I^2}} = \sqrt{2 \overline{I}\,e_0\, \Delta\nu_B}.
}
Šum na izhodu je torej sorazmeren s korenom iz povprečne intenzitete signala in 
s korenom od pasovne širine detekcije oziroma obratno sorazmeren z dolžino 
merjenja. Zapišemo še razmerje signala proti šumu 
\beq
SNR = \frac{\overline{I}}{\sqrt{\overline{\Delta I^2}}}= \frac{\sqrt{\overline{I}}}
{\sqrt{2 e_0\, \Delta\nu_B}}.
\label{SNRs}
\eeq
Po pričakovanjih je to razmerje večje pri večjem povprečnem signalu in pri daljši meritvi.

\begin{remark}
Razmerje signal proti šumu $SNR$ lahko vpeljemo na več načinov. Prvi je ta, ki smo ga 
uporabili mi, pri katerem velja $SNR = \overline{N}/\sigma = \sqrt{\overline{N}}$. 
V tem primeru gre za $SNR$ optične moči oziroma števila fotonov ali elektronov in s 
tem povezanega električnega toka. Lahko pa vpeljemo tudi $SNR_e$ električne moči, ki je, 
zaradi kvadratne zveze med električno močjo in električnim tokom, enak $SNR_e=SNR^2=\overline{N}$.
\end{remark}

\subsection*{Termični šum} 
Termični šum imenujemo tudi Johnsonov\footnote{Švedsko-ameriški elektroinženir in fizik 
John Bertrand Johnson, 1887--1970.} ali Nykvistov\footnote{Švedsko-ameriški elektroinženir
Harry Nyquist, 1889--1976.} ali Johnson-Nyquistov\footnote{Johnson je bil leta 1928 prvi, 
ki je pojav opazoval, Nyquist pa kmalu za eksperimentom podal teoretično razlago.} šum. 
Do njega pride zaradi termično vzbujenega naključnega gibanja elektronov. Ti premiki
na danem uporniku povzročijo majhne kratkotrajne fluktuacije v napetosti, napetost
v povprečju pa seveda ostaja enaka nič.  
Termični šum nastaja samo v uporniških elementih sistema, saj le ti lahko
sprejemajo in oddajajo energijo, v kapacitivnih in induktivnih elementih pa ne.
Izkaže se, da je termični šum najpogosteje omejujoči šum pri detekciji.

Načinov izpeljave termičnega šuma na uporniku je več. Najpogostejša 
je izpeljava na tokokrogu z dvema enakima upornikoma, ki sta v ravnovesju
pri temperaturi $T$\footnote{H. Nyquist, Phys. Rev. {\bf 32}, 110 (1928).}. 
Ko se na uporniku pojavi termična napetost, steče skozi drugi upornik električni 
tok in na njem se porabi elektična moč. Prenos energije z enega upornika
na drug lahko razumemo kot elektromagnetno valovanje, prenešena moč pa
je v ravnovesju enaka porabljeni moči. Naj bo karakterstična impedanca
žic enaka $R$, tako da ne pride do odboja, ampak se val v celoti absorbira.

Zaradi periodičnosti velja za potujoče valove zveza $L = m \lambda$
oziroma $k = m 2\pi/L$. Število elektromagnetnih valov $N$ v frekvenčnem intervalu 
$\Delta\nu_B$ je potem 
\beq
\frac{N}{\Delta\nu_B} = \frac{L}{c},
\eeq
pri čemer je $c$ hitrost valovanja. Posamezne potujoče valove lahko obravnavamo
tudi kot veliko število vzbujenih fotonov z energijo $Nh\nu$. Za njih velja Boltzmannova
porazdelitev, povprečna energija enega vala pa je 
\beq
\overline{E}(\nu) = \frac{h \nu}{e^{h\nu/kT}-1}.
\eeq
Povprečna moč, ki jo prejema drug upornik, je
\beq
\overline{P} = N \frac{\overline{E}}{L/c}= \frac{h \nu \Delta \nu_B}{e^{h\nu/kT}-1}
\approx kT \Delta \nu_B.
\eeq
Moč je po drugi strani enaka
\beq
P = \overline{\Delta I}^2\,R = \frac{\overline{\Delta U^2}}{4R},
\eeq
saj je $I=U/2R$. Od tod sledi
\boxeq{11:termicni}{
\overline{\Delta U^2}  = 4 kTR \Delta \nu_B.
}
Šum lahko zapišemo tudi za tok, če namesto zaporedno vezanega virtualnega 
izvira napetosti vzporedno vežemo virtualni izvor. Dobimo
\boxeq{11:termicni}{
\sqrt{\overline{\Delta I^2}}  = \sqrt{\frac{4 kT\Delta \nu_B}{R}}.
}
Termični šum je torej odvisen od temperature in od upornosti detektorja oziroma
vezja, preko katerega zaznavamo signal. Zmanjšamo ga lahko s hlajenjem upornika
ali s povečanjem upornosti, vendar na ta način zmanjšamo hitrost odziva detektorja. 
Tipične upornosti hitrih detektorjev so tako $R \sim 50~\si{\ohm}$. Termičnega
šuma povsem ne moremo nikoli odpraviti. 

\subsection*{Šum temnega toka} 
Natančna opazovanja pokažejo, da na večini detektorjev zaznamo nek majhen izhodni 
signal tudi v odsotnosti svetlobe. To je temni tok, do katerega pride zaradi
spontanega nastanka para elektron-vrzel ali spontane emisije elektronov 
(glej enačbo~\ref{11:dioda}). Izraza za temni tok tukaj ne bomo izpeljevali,
povejmo le, da je sorazmeren s površino diode in 
eksponentno odvisen od temperature in energijske reže polprevodnika 
\beq
I_0 = j_0\, S\, e^{-E_g/kT}.
\eeq
Zaradi diskretne narave elektronov se -- podobno
kot v primeru diskretnih vpadnih fotonov -- tudi tukaj pojavi šum štetja, le da tukaj 
namesto povprečne vrednosti signala zaradi vpadle svetlobe nastopa temni tok. 
Enačbo (11:sum) zato zapišemo kot 
\boxeq{11:dark}{
\sqrt{\overline{\Delta I^2}} = \sqrt{2 I_0\,e_0\, \Delta\nu_B}.
}
Manjši šum je torej pri detektorjih, ki imajo manjši temni tok, na primer pri silicju. 
Germanij ima v splošnem večji temni tok in zato tudi več šuma temnega toka. Pomembno
vlogo pa ima tudi temperatura, saj v temnem toku nastopa v eksponentu, in lahko 
s hlajenjem šum temnega toka znatno zmanjšamo. 

\subsection*{Šum zaradi sevanja ozadja}
Kot že ime pove, pride do tega šuma zaradi sevanja ozadja pri končni temperaturi. 
Okolico obravnavamo kot črna telesa in spekter njihovega sevanja opisuje Planckov 
zakon (enačba~\ref{eq:Planck}). Z naraščajočo temperaturo telesa se 
spektralni vrh pomika k nižjim valovnim dolžinam in s tem v infrardeče ali celo 
vidno območje. Največji problem predstavlja sevanje ozadja zato pri meritvah v
območju okoli $10$--$30~\si{\micro\meter}$, kjer še telesa pri sobni temperaturi 
znatno sevajo. Detektorjem za infrardečo svetlobo zato pogosto zmanjšamo aperturo 
na najmanjšo možno, poleg tega jih izoliramo od okolice in hladimo. 

Sevanje ozadja je neodvisno od vpadnega signala. Ker detektor ne loči fotonov, ki 
vpadejo nanj kot signal in tistih, ki vpadejo nanj iz ozadja, se prispevek ozadja 
kar prišteje signalu. Šum štetja (enačba~\ref{11:sum}) se tako poveča na
\beq
\sqrt{\overline{\Delta I^2}} = \sqrt{\frac{2 \eta e_0^2\, \Delta\nu_B}{h\nu}\,
\overline{\left( P + P_o \right)}},
\label{11:ozadje}
\eeq
pri čemer $P_o$ označuje moč vpadne svetlobe iz ozadja.

\begin{remark}
 V detektorjih, kjer pride do notranjega ojačevanja (npr. fotopomnoževalka ali plazovna fotodioda), 
 se skupaj s signalom ojača tudi šum. Če se signal ojača za faktor $G$, se za isti faktor
 povečajo tudi šum štetja, šum ozadja in šum temnega toka. Poleg tega pride do ojačenja šuma
 zaradi naključnega povečevanja števila fotonov med pomnoževanjem signala. Tukaj nastopi
 še dodaten faktor, večji od ena, ki je odvisen of snovi, strukture in ojačenja fotodetektorja. 
 Tipična vrednost je okoli 1,5-2, lahko pa doseže vrednosti tudi nad 10.
\end{remark}

\subsection*{Seštevanje šumov}
Spoznali smo, da je več vrst šuma, ki so pri različnih pogojih različno pomembni. 
V splošnem lahko vse prispevke združimo v skupni šum, pri čemer seštevamo kvadrate
odstopanj
\beq
\overline{\Delta I^2} = \overline{\Delta I^2}_{\check{s}tetja} + 
\overline{\Delta I^2}_{termi\check{c}ni} + \overline{\Delta I^2}_{temni} + 
\overline{\Delta I^2}_{ozadje}.
\eeq
Če vstavimo izraze za tokove (enačbe~\ref{11:sum}, \ref{11:termicni}, \ref{11:dark}
in \ref{11:ozadje}), sledi
\boxeq{skupensum}{
\overline{\Delta I^2} = \left( 2 \overline{I}\,e_0 + 2 I_0\,e_0
+ 2 I_o\,e_0 + \frac{4 kT}{R} \right) \Delta\nu_B.
}
Kot že omenjeno, navadno prevlada termični šum nad ostalimi. Izjema je detekcija v
infrardečem območju, kjer pomembno vpliva šum ozadja, in pri zelo nizkih intenzitetah 
vpadne svetlobe, ko pride do izraza šum štejta. 

Če pogledamo izraz za razmerje signala proti šumu
\beq
SNR = \frac{\overline{I}}{\sqrt{\left( 2 \overline{I}\,e_0 + 2 I_0\,e_0
+ 2 I_o\,e_0 + \frac{4 kT}{R} \right) \Delta\nu_B}},
\eeq
vidimo, da so vsi prispevki v imenovalcu neodvisni od intenzitete vpadne svetlobe
razen šuma štetja. Le-ta je pri majhnih intenzitetah majhen in celoten šum 
zato praktično konstanten. V tem primeru $SNR$ narašča kar linearno z intenziteto
vpadne svetlobe. Pri velikih intenzitah šum štetja prevlada nad ostalimi prispevki
in odvisnost $SNR$ od intenzitete postane korenska. 

\begin{definition}
Oceni šum štetja, termični šum in šum temnega toka na silicijevi fotodiodi, če 
nanjo vpada svetloba z valovno dolžino $\lambda=850~\si{\nano\meter}$
in vpadno močjo $P=0,1~\si{\milli\watt}$. Kvantni izkoristek diode je $85~\%$,
spektralna širina $\Delta\nu_B=150~\si{\mega\hertz}$, temni tok $10~\si{\nano\ampere}$,
skupna upornost $50~\si{\ohm}$ in temperatura $300~\si{\kelvin}$. Pokaži, 
da je razmerje signala proti šumu $SNR\sim250$. 
\end{definition}

Pomemben parameter, ki ga pogosto vpeljemo, je $NEP$ ({\it Noise Equivalent Power} -- 
moč, ki ustreza šumu). Gre za vpadno moč svetlobe, ki je po velikosti primerljiva 
s šumom, in zato predstavlja spodnjo mejo še možne detekcije. To se navadno zgodi 
pri zelo nizkih močeh vpadne svetlobe, pri katerih je šum štetja zanemarljiv.
Zapišimo pogoj, pri katerem je $SNR=1$
\beq
NEP\, \frac{e_0}{h \nu} \eta \approx \sqrt{\left(2 I_0\,e_0
+ \frac{4 kT}{R} \right) \Delta\nu_B}.
\eeq
Sledi
\beq
NEP = \frac{h \nu}{\eta e_0}\sqrt{\left(2 I_0\,e_0
+ \frac{4 kT}{R} \right) \Delta\nu_B}.
\label{NEP}
\eeq
\begin{definition}
Izračunaj $NEP$ za primer germanijeve diode pri vpadni svetlobi z valovno dolžino
$\lambda = 1,5~\si{\micro\meter}$ in kvantnim izkoristkom $\eta=0,5$. Temperatura detektorja
je $T=300~\si{\kelvin}$ in temni tok $I_0=15~\si{\micro\ampere}$. Skupna upornost
je $R=2~\si{\kilo\ohm}$, pasovna širina zajemanja svetlobe pa 
$\Delta\nu_B=150~\si{\mega\hertz}$.
\end{definition}

\begin{remark}
Zaradi priročnosti je pogosto podan $NEP$ na koren spektralne širine, saj ta ni 
karakteristična za detektor, ampak je odvisna od časa zajemanja. Podatek, ki 
ga podajo proizvajalci detektorjev, je potem $NEP$, ki je v enotah 
$\si{\watt}/\sqrt{\si{\hertz}}$. Tipične vrednosti so 
$10^{-11}$--$10^{-15}~\si{\watt}/\sqrt{\si{\hertz}}$, pri čemer je najmanjši
za silicijeve fotodiode. 
\end{remark}

Pri zapisu $NEP$ (enačba~\ref{NEP}) smo privzeli, da termični šum in šum temnega
toka prevladata nad šumom štetja signala. Če uspemo ta dva prispevka znatno 
zmanjšati, tako da postane šum štetja signala vodilni člen, 
dosežemo kvanto limito optične detekcije. Takrat je 
\beq
NEP = \frac{2 h\nu \Delta \nu_B}{\eta},
\eeq
kar znaša za zgoraj opisane primere $NEP \sim 10^{-10}\si{\watt}$. Kvantna limita
je z navadnim merjenjem praktično nedosegljiva, saj je treba odpraviti vse ostale izvore 
šuma. En način, kako jo kljub vsemu lahko dosežemo, je s heterodinskim načinom
detekcije, ki ga bomo spoznali v naslednjem razdelku.

\section{Heterodinska detekcija}
Heterodinska detekcija (pogosto imenovana tudi koherentna detekcija) je poseben način
detekcije svetlobe, ki omogoča zaznavanje zelo šibkih signalov. Za razliko od direktne detekcije,
ki smo jo obravnavali do zdaj in pri kateri pride do neposredne zaznave vpadlega fotona, 
gre pri heterodinski detekciji za zaznavanje valovanja z amplitudo in fazo. Pri takem 
pristopu detektor svetlobe osvetlimo hkrati s signalom in z močno referenčno svetlobo, 
katere frekvenca se le malo razlikuje od frekvence signala.
Vpadni signal zapišemo z
\beq
E_s = E_{s0} \cos(\omega_st+\phi),
\eeq
referenčnega pa z
\beq
E_r = E_{r0} \cos(\omega_rt),
\eeq
pri čemer je $E_{r0}$ konstanta. Če sta oba vpadna snopa vzporedna\footnote{Dodaten pogoj je,
da imata isto polarizacijo in čim bolj podobna polmer in ukrivljenost valovnih front.} je intenziteta, 
ki vpada na detektor, enaka
\beq
I \propto |E|^2 = |E_s+E_r|^2 = |E_{s0}|^2 \, \cos^2(\omega_st+\phi)+
|E_{r0}|^2 \, \cos^2(\omega_rt) + 2E_{s0}E_{r0}\, \cos(\omega_st+\phi)\, \cos(\omega_rt).
\eeq
Prva dva člena v izrazu se zelo hitro spreminjata in zato predstavljata zgolj 
izpovprečen konstanten prispevek. Zanimiv je tretji člen, ki ga lahko zapišemo
kot
\beq
E_{s0}E_{r0}\left( \cos(\omega_st+\omega_rt+\phi)+\cos(\omega_st-\omega_rt+\phi)\right).
\eeq
Člen z vsoto obeh frekvenc se izpovpreči, drug člen pa ostane in ga lahko zaznavamo. 
Pri tem smo privzeli, da je razlika frekvenc dovolj majhna, da seže v odzivno območje
detektorja. Poseben primer, ko sta frekvenci povsem enaki, imenujemo homodinski režim 
detekcije. 

Ker je referenčni žarek navadno bistveno močnejši od signalnega, je celotna intenziteta
na detektorju enaka
\beq
|E|^2 \approx \frac{1}{2}|E_{r0}|^2 + E_{s0}E_{r0}\,\cos(\omega_st-\omega_rt+\phi).
\eeq
S tem znatno pridobimo na občutljivosti, saj na detektorju ne zaznavamo več 
kvadrata majhnega signala, ampak majhen signal, pomnožen z velikim referenčnim. 

Poglejmo še razmerje $SNR$ za tak primer detekcije. Največji prispevek k šumu je 
zaradi šuma štetja referenčne svetlobe, saj je le ta praviloma bistveno močnejša od signala
\beq
\sqrt{\overline{\Delta I^2}} = \sqrt{2I_re_0 \Delta\nu_B}=\sqrt{
 e_0^2\Delta\nu_B\frac{\eta E_{r0}^2}{h\nu}},
\eeq
pri čemer smo z $I_r$ označili tok, ki steče zaradi referenčne svetlobe. Singal v tem 
primeru ni več vpadni signal, ampak kombinirani izhod iz detektorja, ki ga zaznavamo
le pri razliki frekvenc $\omega_s-\omega_r$. Sledi
\beq
SNR = \frac{\frac{e_0 \eta}{h \nu} E_{s0}E_{r0}}{\sqrt{
 e_0^2\Delta\nu_B\frac{\eta E_{r0}^2}{h\nu}}} = \sqrt{\frac{\eta}{h \nu \Delta\nu_B}}E_{s0}.
\eeq
Če to primerjamo z vrednostjo $SNR$ pri navadni detekciji (enačba~\ref{SNRs}), vidimo, 
da se razmerje signala proti šumu pri isti pasovni širini izboljša za faktor $\sqrt{2}$
(oziroma še več, če je prisoten še kakšen drug šum).
Ker je pri navadni detekciji težko meriti pri tako majhni pasovni širini, je razmerje
signala proti šumu v primeru heterodinske detekcije zato praviloma znatno večje. Poleg tega
je heterodinski način detekcije neobčutljiv za svetlobo iz ozadja, zato se pogosto 
uporablja za detekcijo svetlobe in infrardečem območju. 
