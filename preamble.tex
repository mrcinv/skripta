
\usepackage[top=3cm,bottom=3cm,left=3.2cm,right=3.2cm,headsep=10pt,a4paper]{geometry} % Page margins

\usepackage{xcolor,lipsum} % Required for specifying colors by name
\definecolor{ocre}{RGB}{0,56,102} 
\definecolor{lightgray}{RGB}{229,229,229} 
% Font Settings
\usepackage{avant} % Use the Avantgarde font for headings
%\usepackage{times} % Use the Times font for headings
\usepackage{mathptmx} % Use the Adobe Times Roman as the default text font together with math symbols from the Sym­bol, Chancery and Com­puter Modern fonts

\usepackage{microtype} % Slightly tweak font spacing for aesthetics
\usepackage[utf8]{inputenc}
\usepackage[T1]{fontenc} % Use 8-bit encoding that has 256 glyphs
\usepackage[unicode]{hyperref}
\usepackage{empheq}
\usepackage{color}
\usepackage{caption}
\usepackage{afterpage}
\newcommand{\boxeq}[2]{\begin{empheq}[box=\colorbox{ocre!30}]{align}\label{#1}#2\end{empheq}}
\newcommand*{\xdash}[1][8em]{\rule[0.5ex]{#1}{0.55pt}}
\usepackage{makeidx}
\makeindex

% MATHS PACKAGE
\usepackage{amsmath,tikz}
\usetikzlibrary{matrix}
\newcommand*{\horzbar}{\rule[0.05ex]{2.5ex}{0.5pt}}
\usepackage{calc}
\usepackage{siunitx}

% VERBATIM PACKAGE
\usepackage{verbatim}
\usepackage{wrapfig}

\input{structure} % Insert the commands.tex file which contains the majority of the structure behind the template

\newcommand{\beq}{\begin{equation}}
\newcommand{\eeq}{\end{equation}}   
\renewcommand{\arraystretch}{1.5}

\newcommand{\usk}{\mbox()}


\usepackage{subfig}
\usepackage{parskip}

\makeatletter
\setlength{\@fptop}{0pt}
\makeatother


%Copy-paste iz stare skripte.. 


\usepackage[tikz]{bclogo}
\usepackage{graphicx}
\usepackage{xcolor}
\definecolor{darkblue}{rgb}{0,0,0.5} 
\usepackage{transparent}
\usepackage{import}

%\usepackage[thinspace,amssymb]{SIunits}




%\renewcommand{\capfont}{\normalfont\small} 

%\newcommand{\stevedeli}[1]{\phantom{.}{\small\begin{bclogo}[noborder = true,couleur=gray!10,logo=\bclambe]{Ste vedeli?}{#1}\end{bclogo}}\phantom{.}}
%\newcommand{\zanimivo}[2]{\phantom{.}{\small\begin{bclogo}[noborder = true,couleur=gray!10,logo=\bclampe]{#1}{#2}\end{bclogo}}\phantom{.}}
%\newcommand{\literatura}[1]{\phantom{.}\begin{bclogo}[couleur=gray!10,arrondi=0.1,logo=\bcbook]{Literatura}{#1}\end{bclogo}\phantom{.}}

%\newcounter{exercise}[chapter]
%\renewcommand{\theexercise}{\thechapter .\arabic{exercise}}

%\newcommand{\naloga}[2]{\refstepcounter{exercise}
%\label{#1}
%\phantom{.}{\small\begin{bclogo}[couleur=blue!30,arrondi=0.1,logo=\bccrayon,ombre=True,couleurOmbre=black!60,barre=snake]{Naloga  \theexercise}{#2}\end{bclogo}}\vspace{1mm}}


%\graphicspath{{/Users/andrej/Documents/Fotonika/skripta/slike/}}
\graphicspath{{./slike/}}

%\usepackage{prettyref}
%\newrefformat{tab}{Tabela\,\ref{#1}}
%\newrefformat{fig}{Slika\,\ref{#1}}
%\newrefformat{eq}{En.\,\textup{\ref{#1}}}

%\usepackage[uline]{hhtensor}


%\newcommand{\matT}[1] {\matr{\boldsymbol{#1}}^{T}}
%\newcommand{\matI}[1] {\matr{\boldsymbol{#1}}^{-1}}
%\newcommand{\mat}[1] {\matr{\boldsymbol{#1}}}
%\newcommand{\ve}[1] {\boldsymbol{#1}}
%\newcommand{\uve}[1] {\widehat{\boldsymbol{#1}}}
%\newcommand{\dd} {\mathrm{d}}
%\newcommand{\Tr}{\textrm{Tr}}

%\newcommand{\tagarray}{%
%\mbox{}\refstepcounter{equation}%
%$(\theequation)$%
%}

%\renewcommand\Re{\operatorname{Re}}
%\renewcommand\Im{\operatorname{Im}}
